\documentclass[12pt,a4paper,bibliography=totocnumbered,listof=totocnumbered]{scrartcl}
\usepackage[ngerman]{babel}
\usepackage[utf8]{inputenc}
\usepackage{amsmath}
\usepackage{amsfonts}
\usepackage{amssymb}
\usepackage{graphicx}
\usepackage{fancyhdr}
\usepackage{tabularx}
\usepackage{geometry}
\usepackage{setspace}
\usepackage[right]{eurosym}
\usepackage[printonlyused,footnote]{acronym}
\usepackage{subfig}
\usepackage{floatflt}
\usepackage[usenames,dvipsnames]{color}
\usepackage{colortbl}
\usepackage{paralist}
\usepackage{array}
\usepackage{titlesec}
\usepackage{parskip}
\usepackage[right]{eurosym}
\usepackage{picins}
\usepackage{textcomp} %Eurozeichen und andere Symbole
\usepackage{xspace} %Leerzeichen nach LaTeX -Befehlen
\usepackage[subfigure,titles]{tocloft}
\usepackage[pdfpagelabels=true]{hyperref}
\usepackage[babel,german=quotes]{csquotes}
\usepackage{romannum}

\usepackage[
	backend=biber,
	style=alphabetic,
	citestyle=alphabetic,
	sorting=nty,
	doi=true,url=true
]{biblatex}
\bibliography{bib}

\usepackage{listings}
\lstset{basicstyle=\footnotesize, captionpos=b, breaklines=true, showstringspaces=false, tabsize=2, frame=lines, numbers=left, numberstyle=\tiny, xleftmargin=2em, framexleftmargin=2em}
\makeatletter
\def\l@lstlisting#1#2{\@dottedtocline{1}{0em}{1em}{\hspace{1,5em} Lst. #1}{#2}}
\makeatother

\geometry{a4paper, top=27mm, left=30mm, right=20mm, bottom=35mm, headsep=10mm, footskip=12mm}

\hypersetup{unicode=false, pdftoolbar=true, pdfmenubar=true, pdffitwindow=false, pdfstartview={FitH},
	pdftitle={Abschlussarbeit},
	pdfauthor={Daniel Brettschneider},
	pdfsubject={Abschlussarbeit},
	pdfcreator={\LaTeX\ with package \flqq hyperref\frqq},
	pdfproducer={pdfTeX \the\pdftexversion.\pdftexrevision},
	pdfkeywords={Abschlussarbeit},
	pdfnewwindow=true,
	colorlinks=true,linkcolor=black,citecolor=black,filecolor=magenta,urlcolor=black}
\pdfinfo{/CreationDate (D:20110620133321)}

% Hurenkinder und Schusterjungen verhindern
\clubpenalty10000
\widowpenalty10000
\displaywidowpenalty=10000

% Benutzerdefinierte Befehle für späteren Werteabruf
\def\title#1{\gdef\@title{#1}\gdef\THETITLE{#1}}
\def\author#1{\gdef\@author{#1}\gdef\THEAUTHOR{#1}}
\newcommand{\authoremail}{\href{mailto:\email}{\email}}

% Wertezuweisung in Variablen, damit diese nur an einer Stelle geändert werden müssen
\title{Entwicklung eines Android Usability Testing Frameworks}
\author{Albert Hoffmann}
\newcommand{\email}{albert.hoffmann@hs-osnabrueck.de}
\newcommand{\pruefer}{Prof. Michaela Ramm, M.A.}

\begin{document}

\titlespacing{\section}{0pt}{12pt plus 4pt minus 2pt}{-6pt plus 2pt minus 2pt}

% Kopf- und Fusszeile
\renewcommand{\sectionmark}[1]{\markright{#1}}
\renewcommand{\leftmark}{\rightmark}
\pagestyle{fancy}
\lhead{}
\chead{}
\rhead{\thesection\space\contentsname}
\lfoot{\THETITLE}
\cfoot{}
\rfoot{\thepage}
\renewcommand{\headrulewidth}{0.4pt}
\renewcommand{\footrulewidth}{0.4pt}

% Vorspann
\renewcommand{\thesection}{\Roman{section}}
\renewcommand{\theHsection}{\Roman{section}}
\pagenumbering{Roman}

% ----------------------------------------------------------------------------------------------------------
% Titelseite
% ----------------------------------------------------------------------------------------------------------


\thispagestyle{empty}
\begin{center}
	\includegraphics[scale=1]{img/hs_os}\\
	\vspace*{2cm}
	\Large
	\textbf{Fakultät}\\
	\textbf{Ingenieurwissenschaften und Informatik}\\
	\vspace*{2cm}
	\Huge
	\textbf{Projektdokumentation}\\
	\vspace*{0.5cm}
	\large
	für das \\
	\textbf{Masterprojekt}\\  
	mit dem Thema\\
	\vspace*{1cm}
	\textbf{\THETITLE}\\
	\vspace*{2cm}
	
	\vfill
	\normalsize
	\newcolumntype{x}[1]{>{\raggedleft\arraybackslash\hspace{0pt}}p{#1}}
	\begin{tabular}{x{6cm}p{7.5cm}}
		\rule{0mm}{5ex}\textbf{Autor:} & \THEAUTHOR \newline 
		\authoremail \\ 
		\rule{0mm}{5ex}\textbf{Prüfer:} & \pruefer \\ 
		\rule{0mm}{5ex}\textbf{Abgabedatum:} & 06.10.2014 \\ 
	\end{tabular} 
\end{center}
\pagebreak


\begin{abstract}

\noindent\textbf{Zusammenfassung:}\\ \\
Die vorliegende Ausarbeitung wurde in \LaTeX verfasst und ist eine gemeinsame Arbeit von Albert Hoffmann und Oliver Erxleben an der Hochschule Osnabrück / University of Applied Sciences im Masterstudiengang Informatik - Mobile und Verteilte Anwendungen des Fachbereichs Ingenieurswissenschaften und Informatik für das Modul Mensch-Maschine-Kommunikation im Wintersemester 2013/14. 

Die Arbeit dokumentiert die Entwicklung einer Programmierbibliothek zum Test der Usability von Mobilanwendungen für das Android-Betriebssystem.

Das Ziel ist es, eine Bibliothek für das Android Betriebssystem zu schreiben, die Entwickler einfach in ihre Anwendungen integrieren können und mit der sich Usability Untersuchungen durchführen lassen.

Zur Verwaltung wird für Entwickler/Testleiter eine webbasierte Oberfläche bereitgestellt, über die Test und Aufgaben verwaltet werden können. 
Das Ergebnis ist ein Video, welches das Kamerabild der Gerätefrontkamera und den Bildschirminhalt des Testgerätes enthält. Diese Video kann über die Verwaltungssoftware, nach Abschluss der Tests, Angezeigt werden.

Für die Kommunikation wird außerdem eine \ac{API} entwickelt, welche von der Verwaltungsoberfläche und der Android Bibliothek genutzt wird.
Alle Komponenten werden als Prototypen entwickelt und sind eher als Proof-of-Concept anzusehen, entsprechend eignen sich die frühen Versionen weniger für den Produktiveinsatz.

\end{abstract}
% Verzeichnisse
\input{verzeichnisse}
% ----------------------------------------------------------------------------------------------------------
% Abkürzungen
% ----------------------------------------------------------------------------------------------------------
\section{Abkürzungsverzeichnis}
%TODO: längste Abkürzung steht in eckigen Klammern
\begin{acronym}[HTTP]
\setlength{\itemsep}{-\parsep} % geringerer Zeilenabstand
\acro{OSGi}{Open Service Gateway initiative} %TODO entfernen
\acro{HTTP}{Hypertext Transfer Protocol}
\acro{SSL}{Secure Sockets Layer}
\acro{TLS}{Transport Layer Security}
\acro{UI}{User Interface}
\acro{JSON}{JavaScript Object Notation}
\acro{UX}{User Experience}
\acro{SDK}{Software Development Kit}
\acro{IDE}{Integrated Development Environment}
\acro{API}{Application Programming Interface}
\acro{NPM}{Node Package Manager}
\acro{VM}{Virtuelle Maschine}
\acro{SSH}{Secure Shell}
\acro{JAR}{Java Archive File}
\acro{CSS}{Cascading Style Sheets}
\acro{SQL}{Structured Query Language}
\acro{ER}{Entity Relationship}
\acro{REST}{Representational State Transfer}
\acro{URL}{Uniform Resource Locator}
\end{acronym}


% ----------------------------------------------------------------------------------------------------------
% Inhalt
% ----------------------------------------------------------------------------------------------------------
% Abstände Überschrift
\titlespacing{\section}{0pt}{12pt plus 4pt minus 2pt}{-6pt plus 2pt minus 2pt}
\titlespacing{\subsection}{0pt}{12pt plus 4pt minus 2pt}{-6pt plus 2pt minus 2pt}
\titlespacing{\subsubsection}{0pt}{12pt plus 4pt minus 2pt}{-6pt plus 2pt minus 2pt}

% Kopfzeile
\renewcommand{\sectionmark}[1]{\markright{#1}}
\renewcommand{\subsectionmark}[1]{}
\renewcommand{\subsubsectionmark}[1]{}
\lhead{Kapitel \thesection}
\rhead{\rightmark}

\onehalfspacing
\renewcommand{\thesection}{\arabic{section}}
\renewcommand{\theHsection}{\arabic{section}}
\setcounter{section}{0}
\pagenumbering{arabic}
\setcounter{page}{1}


\section{Einleitung}
\subsection{Bezug der Programmquellen und Dokumentation}
Die Programmquellen der entwickelten, bzw. weiterentwickelten Software ist in dem folgenden \emph{GitHub}-Repository abrufbar:
\begin{center}
	\url{https://github.com/Chromdioxyde/hipsterbility}
\end{center}
Als Versionsverwaltung wird \emph{Git}\footnote{Git Webseite: \url{http://git-scm.com/}} eingesetzt.
Mit dem Befehl
\\
 \texttt{git clone https://github.com/Chromdioxyde/hipsterbility.git} kann eine lokale Kopie erstellt werden.
Alternativ kann der Inhalt des Repositories auf der o.~g. Webseite auch als ZIP-Datei heruntergeladen werden.
Die Quellen der Programme liegen im \emph{IntelliJ IDEA}\footnote{IntelliJ IDEA Website: \url{https://www.jetbrains.com/idea/}} Projektformat vor.
Für die Entwicklung wurde \emph{IntelliJ IDEA Ultimate 13.1.5} verwendet.

\subsection{Motivation}
Bei begrenztem Budget, z.~B. von Studierenden sind umfangreiche Labortests oder das Beauftragen von Dienstleistern oft nicht möglich.

Ein weiterer Faktor ist auch die Zielgruppe für die entwickelte Anwendung.
Bei räumlich verteilten Anwendern sind Labortests durch den logistischen Aufwand nur schwer möglich und auch die finanzielle Belastung steigt durch Transportkosten und evtl. mobiles Equipment.

Deshalb wurde im Wintersemester 2013/2014, zusammen mit Oliver Erxleben, ein Projekt durchgeführt, mit dem Ziel Usability-Tests mit minimalem Aufwand für den Teilnehmer durchzuführen (siehe Abschnitt \ref{subsec:stand_bei_projektbeginn}).

Die damals gesammelten Erkenntnisse und Erfahrungen sollen nun vertieft werden, mit dem Ziel die bestehenden Prototypen zu einem nutzbaren Framework weiter zu entwickeln.


\subsection{Zielsetzung}
Das Ziel ist es, ein Usability-Testing Framework oder Toolkit zu entwickeln, mit dem auch einzelne Entwickler oder kleine Teams ihre Anwendungen und Prototypen mit der Hilfe von Testbenutzern testen können.
Für den eigentlichen Testablauf beim Benutzers soll nur ein Android basiertes Gerät und eine Internetverbindung benötigt werden.
Ziel dabei ist es, das Testen möglichst einfach zu gestalten, ohne Labor und zusätzliches Equipment.

Auf der Seite des Testleiters soll nur ein PC oder Notebook benötigt werden, ohne zusätzliche Hardware.
Die Serverkomponente wurde ausgelagert, für den Fall, dass ein Server zur Verfügung steht, bzw. z.~B. von einer Bildungseinrichtung bereitgestellt wird.
\subsection{Problemstellung}
Wie in Abschnitt \ref{sec:stand_der_technik} näher erläutert wird, gibt es zwar kommerzielle Anbieter für das Remote Usability Testing, jedoch beschränkt sich deren Angebot entweder auf mobile Websites oder ist mit hohen Kosten verbunden.
Auch die verfügbaren Werkzeuge bieten keine zufriedenstellende Lösung für das entfernte Testen.

Weiterhin besteht zwar die Möglichkeit den Bildschirm von Android Geräten aufzuzeichnen und Interaktionen darzustellen, jedoch wird dafür entweder ein zusätzlicher PC benötigt, oder es besteht keine einfache Möglichkeit diese Aufnahmen auswertbar zu machen und weiterzugeben.

Neben den Bildschirmaufgaben müssen auch Aufgaben und Anweisungen an den Benutzer übermittelt, sowie zusätzliche Metadaten gesammelt werden, um später statistische Aussagen machen zu können.

Für einen Teil dieser Probleme bestehen bereits Lösungsansätze, auf welche im nächsten Abschnitt verwiesen wird.
\subsection{Stand bei Projektbeginn\label{subsec:stand_bei_projektbeginn}}
Dieses Projekt basiert auf einer Arbeit, welche zusammen mit Oliver Erxleben (\href{mailto:oliver.erxleben@hs-osnabrueck.de}{oliver.erxleben@hs-osnabrueck.de}) im Modul \emph{Mensch-Maschine-Kommunikation} (Wintersemester 2013/2014) des Informatik--Masterstudiengangs \emph{Verteilte und mobile Anwendungen} erstellt wurde.
Die Ergebnisse zum Stand der Abgabe können aus dem Git-Repository von Oliver Erxleben unter der folgenden \ac{URL} abgerufen werden:

\url{https://github.com/olivererxleben/hipsterbility/releases/tag/v1.0_semester_final}

Der Vollständigkeit halber wird der Stand der vorausgehenden Arbeit an dieser Stelle kurz wiedergegeben.
Es wurde ein Prototyp erstellt, welcher aus einer Android Bibliothek und einem kleinen \ac{REST}-Server besteht.
Zum Testen der Bibliothek wurde weiterhin eine kleine Android Applikation erstellt.
Insgesamt wurde folgender Funktionsumfang skizziert:
\begin{compactitem}
	\item Android 4.x Bibliothek
	\begin{itemize}
		\item Abrufen von Testsitzungen und Testaufgaben vom Server.
		\item Benutzeroberfläche zum Auswählen und Starten von Testsitzungen.
		\item Aufzeichnen des Kamerabildes der Frontkamera mit Ton, nach dem Starten einer Testsitzung.
		\item Erstellen von Screenshots der Applikation bei Benutzereingaben auf dem Touchscreen und Visualisierung der Eingaben auf dem Screenshot.
		\item Hochladen der gesammelten Daten zum Server.
	\end{itemize}
	\item Node.js basierter Server mit REST-Schnittstelle und Webinterface
	\begin{itemize}
		\item Bereitstellen von Testsitzungen und -aufgaben.
		\item Rudimentäres Benutzermanagement.
		\item Weboberfläche zum Erstellen von Testsitzungen und -aufgaben
		\item Speichern der hochgeladenen Daten in einer Datenbank und im Dateisystem.
		\item Aufbereiten der Screenshots und des Videos von der Frontkamera in ein Ergebnisvideo für die Auswertung.
	\end{itemize}
\end{compactitem}

Die Idee und die ursprüngliche Planung stammen von Oliver Erxleben, welcher auch für die Serverkomponente zuständig war.

Da viele Funktionen in der Implementierung nur skizziert wurden eignet sich der Prototyp nicht für die Verwendung in einer produktiven Umgebung.

Folgende Komponenten wurden aus der vorhergehenden Arbeit übernommen und weiterentwickelt:
\begin{compactitem}
	\item das grundlegende Konzept,
	\item Teile des Datenmodells,
	\item und Quellen der Android Bibliothek und Testapplikation.
\end{compactitem}

Der Server wird auf der Basis von \emph{Java EE 7} neu aufgebaut.
%TODO: Querverweis

\section{Stand der Technik}
\subsection{Mobile Usability Testing Methoden}
Usability-Tests können grob in zwei räumliche Kategorien eingeteilt werden, Labortests und Feldtests, welche jeweils moderiert oder unmoderiert stattfinden können.

Ein einfacher Testaufbau für Labortests lässt sich z.B. mit Hilfe eines Computers mit angeschlossener, externer Kamera realisiert werden. 
Vorausgesetzt werden auch entsprechende Räumlichkeiten und Benutzer, welche die Tests durchführen.
Die, an den Computer angeschlossene Kamera, dient zum Aufzeichnen und Dokumentieren des Bildschirms eines mobilen Gerätes auf dem getestet wird.
Über die Kamera werden auch die Benutzereingaben dokumentiert. \cite[Vgl.][]{Budiu.2014}

\subsubsection{Thinking Aloud\label{subsec:thinking_aloud}}
\subsection{Mobile Usability Testing Anbieter\label{subsec:usability_testing_anbieter}}
Neben Methoden und Werkzeugen zum Messen und Testen von Software-Usability gibt es auch Dienstleister, deren Geschäftsmodell auf Usability-Tests und Untersuchungen basiert.
Hier werden einige Anbieter solcher Dienstleister kurz vorgestellt und das Angebot kurz beschrieben.
Die Auswahl und Beschreibung beschränkt sich auf Anbieter, die das Testen von mobile Applikationen anbieten.

\subsubsection{UserTesting\label{subsubsec:usertesting}}
Der Dienstleister \emph{UserTesting} (\url{http://www.usertesting.com/}) bietet, unter anderem, Usability-Studien an, welche von Testteilnehmern durchlaufen werden.
Das Angebot beinhaltet auch das Testen von mobilen Webseiten und Applikationen. \cite[Vgl.][]{UserTesting.2014c}

Der Kunden/Auftraggeber stellt Testaufgaben zusammen, welche die Testpersonen erfüllen sollen.
Es gibt die Möglichkeit Geräteklasse, Anzahl der Testteilnehmer und eine Zielgruppe festzulegen.
Das Ergebnis der in Auftrag gegebenen Studie kann, laut Anbieter, schon nach ca. einer Stunde vorliegen und umfasst Videos von Testteilnehmern währen der Tests, schriftliche Antworten und die Möglichkeit der anschließenden Befragung der Teilnehmer. \cite[Vgl.][]{UserTesting.2014b}

Das Angebot zum Testen von mobilen Applikationen umfasst die Plattformen Android und iOS.
Pro Plattform werden als Testgeräte Tablets und Smartphones angeboten, welche bei der Spezifikation einer Teststudie angegeben werden. 
Auch hier werden Zielgruppen ausgewählt und Testaufgaben spezifiziert.
Das Testergebnis umfasst auch eine Videoaufzeichnung vom Test, auf dem der Gerätebildschirm sichtbar ist und der Benutzer Aussagen nach der \emph{Thinking Aloud} Methode (siehe Abschnitt \ref{subsec:thinking_aloud}) tätigt.
Die pauschalen Kosten für kleine Studien belaufen sich auf \$49 pro Testbenutzer.
\cite[Vgl.][]{UserTesting.2014}

\subsubsection{UserZoom \label{subsubsec:userzoom}}
Das \emph{UserZoom} (\url{http://www.userzoom.de}) Angebot vereint Werkzeuge und Dienstleistungen.
Letztere sind vorwiegend unterstützender Natur und umfassen z.B. Beratung, Rekrutierung von Teilnehmern und den Kunden-Support \cite[vgl.][]{UserZoomGmbH.2013b}.

Das Angebot bzgl. des Mobile Testing umfasst eine mobile Applikation für die iOS und Android Plattformen, welche sich zum Testen von Webseiten und webbasierten App-Prototypen eignet.
Das Verfahren wird als \enquote{[...] Remote Unmoderated Mobile Usability Testing [...]} \cite{UserZoomGmbH.2013c} bezeichnet und erlaubt ein standortunabhängiges Testen.
Angemeldete Testpersonen werden in Zielgruppen eingeteilt und per E-Mail zu Studien und Befragungen eingeladen.
Das eigentliche Testen wird über eine native mobile Applikation durchgeführt, in welcher Fragen beantwortet und Aufgaben erfüllt werden sollen.
Mit der Applikation lassen sich keine nativen Applikationen testen.
\cite[Vgl.][]{UserZoomGmbH.2013c}

Im Gegensatz zu \emph{UserTesting} gibt \emph{UserZoom} keine Pauschalpreise an.
Ein Rechenbeispiel auf der Webseite des Unternehmens vergleicht Remote Usability Testing mit Labortests.
Dabei werden Kosten von bis zu 120~\texteuro\xspace pro Testbenutzer in Labortests und ca. 10~\texteuro\xspace für Remote-Testbenutzer berechnet.
Die Kostengegenüberstellung am Ende des Artikels, welche eine hypothetische Ersparnis von 40~\% berechnet lässt mit 12\,440~\texteuro\xspace zu 7600~\texteuro\xspace darauf schließen, dass sich das Angebot eher an Unternehmen richtet, als an einzelne Entwickler oder kleine Teams.
\cite[Vgl.][]{UserZoomGmbH.2013d}
\subsection{Mobile Usability Testing Werkzeuge}
\subsection{Technologien und Werkzeuge}

\subsubsection{Android Bildschirmaufzeichnung}
Eine der Neuerungen, welche mit Android 4.4 \enquote{KitKat} eingeführt wurde, ist die Möglichkeit den Bildschirminhalt in Form eines Videos aufzuzeichnen.
Die Aufzeichnung wird über einen PC gestartet, der mit dem Android Gerät über die \ac{ADB} \cite{AndroidDevelopers.2014b} per \ac{USB}-Kabel oder kabellos verbunden ist.
Die Aufzeichnung wird mit dem Befehl \texttt{adb shell screenrecord} gestartet werden.
Soll ein Teil einer Applikation nicht aufgezeichnet werden, so kann der Entwickler die Aufnahme mit \texttt{SurfaceView.setSecure()} verhindern.
Die Aufzeichnung wird im MP4"~Format auf dem Gerät gespeichert und lässt sich von dort abrufen, z.~B. mit der \ac{ADB} oder über eine \ac{USB} \ac{MTP}.
\cite[Vgl.][]{AndroidDevelopers.2014}
\begin{wrapfigure}{l}{.3\textwidth}
	\centering
	\includegraphics[width=\linewidth]{img/screenshot_omnirom_power_menu}
	\caption{Screenshot des OmniROM Power-Menüs.}
	\label{fig:screenshot_omnirom_power_menu}
\end{wrapfigure}
\ac{ADB} ist ein Debugging-Werkzeug, welches als Bestandteil des Android \ac{SDK} \cite{AndroidOpenSourceProject.2014c} ausgeliefert wird.
Mit ihm lassen sich Befehle direkt auf Shell des Geräts ausführen.
Vor Android 4.4 war es bereits möglich mit dem Befehl \texttt{adb shell screencap -p /sdcard/<name>.png} Bildschirmfotos zu erstellen \cite[vgl.][]{RandomStuff.2013}.
Diese Möglichkeit besteht seit Android 4.0 \enquote{Ice Cream Sandwich} (ICS).

Die beiden genannten Möglichkeiten für Aufzeichnungen des Bildschirminhalts benötigen jedoch einen per \ac{ADB} verbundenen PC.

Um diese Befehle direkt auf dem Gerät aus einer Applikation heraus auszuführen, wird der Root-Zugriff auf dem Gerät benötigt \cite[vgl.][22]{Erxleben.2014}.

Alternativ bieten einige Geräte und sog. \enquote{Custom-ROMs} die Möglichkeit Bildschirmfotos und -videos direkt aus einem Menü heraus oder per Tastenkombination zu erstellen. 
Abbildung \ref{fig:screenshot_omnirom_power_menu} zeigt beispielhaft das Power-Menü von \emph{OmniROM}\footnote{OmniROM Webseite: \url{http://omnirom.org/}}.

Eine Möglichkeit um den Bildschirminhalt auf einem zusätzlichen Display anzuzeigen wird im nächsten Abschnitt kurz erläutert.

\subsubsection{Chromecast Screen Mirroring}
Der Google \emph{Chromecast} ist ein Streaming-Client, welcher z.~B. an Monitoren oder Fernsehern angeschlossen werden kann.
Nach der ersten Einrichtung kann er u.~a. über Android Geräte im selben Netzwerk gesteuert werden.
Applikationen mit \emph{Chromecast} Unterstützung können Medien auf diesem Weg Inhalte direkt auf einem Bildschirm darstellen. \cite[Vgl.][]{Google.2014}
Neben dem Streaming von Medieninhalten aus dem Internet unterstützt die aktuelle \emph{Chromecast} Applikation für Android \cite{GoogleInc..2014} auch das Übertragen des Bildschirminhalts mit sehr geringer Latenz.
Diese Funktion ist allerdings nicht für alle Geräte verfügbar.
Eine Aufzeichnung der Übertragung wird aktuell nicht unterstützt.

Auch wenn es andere Lösungen gibt, um den Bildschirminhalt eines mobilen Geräts zu übertragen, \emph{Chromecast} ist mit 35~\texteuro\xspace \footnote{Preisangabe des Herstellers, Stand: Oktober 2014. Webseite: \url{https://www.google.de/chrome/devices/chromecast/}} einer der preiswertesten Wege, bei dem nur ein Fernseher oder Monitor benötigt wird.
Besonders spannend sind die Möglichkeiten für Labortests, denn die Kamera, welche den Bildschirm abfilmt könnte entfallen und der Testbenutzer kann sich frei im Raum, bzw. in der Reichweite des kabellosen Netzwerks, bewegen.

\section{Konzept}
\subsection{Anforderungen}\label{subsec:anforderungen}
An das Projekt ergeben sich die folgenden Anforderungen:
\begin{compactitem}
	\item Einfache Benutzung
	\item Einfache Integration in bestehende Apps
	
\end{compactitem}
An das Projekt ergeben sich die folgenden Anforderungen:
\begin{compactitem}
	\item  
\end{compactitem}

\section{Realisierung des Servers}
\subsection{RESTFul Webservice}
Um die Android Bibliothek und den Testleiter-Client anzubinden wird eine \ac{REST}-\ac{API} verwendet.
Da der \emph{Glassfish} \ac{AS} die Grundlage für die Serverkomponente darstellt, kommt die \ac{JAX-RS} 2.0 Implementierung \emph{Jersey} (\url{https://jersey.java.net/}) zum Einsatz, welche beim \ac{AS} mitgeliefert wird.
Als Referenz für die Implementierung der Schnittstelle und den Aufbau der Ressourcen dient \citetitle{Burke.2014} von \citeauthor{Burke.2014} \cite{Burke.2014}.

Eine möglichst lose Kopplung der verschiedenen Klassen wird durch die Verwendung von Interfaces und Dependency Injection erreicht.
Ein gutes Beispiel dafür zeigt Robert Leggett mit seinem GitHub Projekt \emph{jersey\_restful\_webservice}\footnote{GitHub Repository: \url{https://github.com/Robert-Leggett/jersey_restful_webservice}}, welches vom Ressourcenaufbau eher dem \ac{REST}-\ac{RPC} Schema folgt.

Aktuell wird nur das \ac{JSON} Datenformat unterstützt.
Für das (Un--)Marshalling wird serverseitig die \emph{Jackson} (\url{https://github.com/FasterXML/jackson}) Bibliothek verwendet.

Eine Implementierung des \ac{HATEOAS} Prinzips \cite[vgl.][11-13]{Burke.2014} wurde nicht vorgenommen, da die entwickelte \ac{API} nicht öffentlich zugänglich sein wird.
Außerdem wird sie aktuell nur von den eigenen Clients genutzt, welchen der Aufbau der \ac{API} nd Ressourcen bekannt ist.
Ein detaillierte Beschreibung der Ressourcen mit Pfaden und \ac{HTTP}"~Methoden befindet sich in Abschnitt \ref{subsubsec:rest_ressourcen}.

\subsubsection{Java REST-API mit JAX-RS Annotationen und CDI}

\subsubsection{Implementierte Ressourcen\label{subsubsec:rest_ressourcen}}
Die Ressourcen werden als Gruppen von Objekten angesehen und es werden Nomen im Plural verwendet \cite[vgl.][20\psqq]{Burke.2014} um die Benennung einheitlich zu gestalten (z.B. \texttt{users, sessions, devices}).
Dies weicht vom Datenbankmodell ab, da dort die Tabellen im Singular benannt sind (siehe Abschnitt \ref{subsec:datenbank}).

Wie auch bei den \acp{DAO} in der Persistenzschicht der Serveranwendung wurden die vier grundlegenden Datenoperationen \ac{CRUD} implementiert.
Die Zuordnung zwischen Operation und \ac{HTTP}"~Methoden bzw. \ac{SQL}"~Befehlen ist in Tabelle \ref{tbl:crud} zu sehen.

\begin{minipage}[t]{\textwidth}
	\centering
	\begin{tabu}{|X|>{\ttfamily}X|>{\ttfamily}X|}
	\rowfont[l]{\normalfont\bfseries} 
		\hline Operation & HTTP & SQL \\ 
		\hline Create & POST & INSERT \\ 
		\hline Read / Retrieve & GET & SELECT \\ 
		\hline Update / Modify & PUT & UPDATE \\ 
		\hline Delete & DELETE & DELETE \\ 
		\hline 
	\end{tabu}
	\captionof{table}{Zuweisung der CRUD Operationen und HTTP-Methoden und SQL-Befehlen.}
	\label{tbl:crud}
\end{minipage}

\subsubsection{Aufbau von Ressourcen und Zugriffsrechte}
Die Ressourcen sind flach gehalten.
Je nach Benutzerrolle führen Aufrufe der Methoden zu verschiedenen Ergebnissen.
Bei Ressourcen mit Benutzerbezug (z.~B. Geräte und Testsitzungen) bekommt ein angemeldeter Benutzer in der Rolle \texttt{USER} nur Objekte, welche mit ihm in Relation stehen.
Dadurch wird sichergestellt, dass einfache Benutzer nicht auf Daten anderer Benutzer zugreifen können.
Benutzer in der Rolle \texttt{ADMIN} können hingegen auf alle Objekte in einer Ressource zugreifen.

Das erstellen von Objekten mit Benutzerbezug über die \texttt{POST}"~Methode erfordert in der Rolle \texttt{ADMIN} zusätzlich zu dem zu erstellenden Objekt auch noch die \texttt{ID} des Benutzers, für welchen diese erstellt werden soll, sofern die Objekte keine bidirektionale Beziehung haben und sich der Bezug daraus herstellen lässt. 

Nachfolgend werden die Resourcen mit den implementierten \ac{HTTP}"~Methoden aufgelistet.
Die Pfadangabe erfolgt relativ zur Basis"~\ac{URL}.
Angaben in geschweiften Klammern sind Pfad-Parameter, welchen im Aufruf ein Wert zugewiesen ist.
%TODO: fehlende Ressourcen nachpflegen


\begin{minipage}[t]{\textwidth}
	\centering
	\begin{tabu}{>{\ttfamily}X[2]X[1]>{\ttfamily}X[3]>{\small}X[4]}
		\rowfont[l]{\normalfont\bfseries\normalsize}
		Rollen & Methode & Pfad &   Beschreibung\\
		USER & GET & /users &   Daten des angemeldeten Benutzers\\ 
		ADMIN & GET & /users &  Liste aller Benutzer\\
		USER & PUT & /users & Bestätigung oder Fehlermeldung\\
		ADMIN & PUT & /users/\{id\} & Bestätigung oder Fehlermeldung\\ 
		ALL, ADMIN & POST & /users &  ID des neu erstellten Benutzers\\
		ADMIN & GET & /users/\{id\} & Ausgabe des Benutzers mit der ID \texttt{id}\\
		ADMIN & DELETE & /users/\{id\} & Löschen des Benutzers mit der ID \texttt{id}\\
		\\
		USER & GET & /devices & Liste mit Geräten des Benutzers\\
		ADMIN & GET & /devices & Liste mit allen Geräten\\
		ADMIN & GET & /devices/\{id\} & Gerät mit der ID \texttt{id}\\
		
		ADMIN & DELETE & /devices/\{id\} & Gerät mit der ID \texttt{id}\\
		ADMIN & PUT & /devices/\{id\} & Gerät mit der ID \texttt{id}\\
	\end{tabu}
	\captionof{table}{REST"~Ressourcen mit HTTP"~Methoden, Pfad, Benutzerrolle und Beschreibung.}
	\label{tbl:rest_ressourcen}
\end{minipage}



\section{Realisierung der Android Bibliothek}
\section{Realisierung des Testleiter-Clients}


\section{Ergebnisse}

\section{Fazit}
\subsection{Weiterentwicklung}
\subsubsection{Integrierte Blickverfolgung}
Projekte wie \emph{Opengazer} \cite{TheInferenceGroup..2009} zeigen eindrucksvoll, dass Eye-Tracking bzw. Gaze-Tracking auch ohne spezielles Equipment möglich ist.
Mit Hilfe einer einfachen Webcam gelang dem Entwicklerteam das Verfolgen von Augenbewegungen und die Visualisierung des Fokuspunktes auf dem Computerdisplay.
Die quelloffene Software benötigt für die korrekte Funktion das Markieren von Referenzpunkten im Gesicht auf dem Kamerabild, sowie eine Kalibrierung des Blickes mit Referenzpunkten auf dem Display.

Ein schwerwiegender Nachteil der aktuellen Version ist allerdings, dass der Kopf möglichst still gehalten werden und die Webcam fixiert werden muss.
Selbst wenn die Software auf mobile Betriebssysteme portiert und ausgeführt werden könnte, wäre es nur schwer möglich die zuvor genannten Bedingungen zu erfüllen.
Das Testgerät müsste fixiert werden und auch die Testperson dürfte sich beim Test möglichst wenig bewegen.
Die Qualität der Messungen wäre auch bei perfekten Bedingungen fraglich, da schon kleine Messungenauigkeiten den Wert der Messpunkte stark reduzieren würden, da die Displays von mobilen Geräten sehr viel kleiner sind als bei Desktop-PCs oder Notebooks.

\subsubsection{Blickerkennung mit zusätzlicher Hardware}
Eine alternative zur Blickverfolgung mit der integrierten Kamera von mobilen Geräten könnte \emph{The Eye Tribe} \cite{TheEyeTribe.2014} bieten.
Diese bieten einen kompakten Eye-Tracker für \$99 an.
Die Hardware ist mit 1,9 x 20 x 1,9 cm Ausmaßen sehr kompakt und mit 70 g auch leicht genug für den mobilen Einsatz, zumindest an Tablets.
Aktuell ist die Software nur unter Microsoft Windows lauffähig, eine Android Version befindet sich, laut Herstellerwebsite, in der Entwicklung.

Blickverfolgung auf einem Smartphone ohne externe Hardware könnte schon in naher Zukunft möglich sein.
Durch die Verwendung von mehreren Kameras in der Gerätefront kann das \emph{Amazon Fire Phone} \cite{amazon.de.2014} schon jetzt die Perspektive des Bildschirminhalts dynamisch an den Betrachtungswinkel des Benutzers anpassen.
Zwar wird das Eye-Tracking vom aktuellen \emph{Fire Phone \ac{SDK}} (noch) nicht unterstützt, lediglich das Head-Tracking, jedoch ist das Smartphone erst seit kurzem in den U.S.A. verfügbar und der Verkaufsstart in Deutschland steht noch aus (stand September 2014).
Welche Möglichkeiten die zusätzlichen Kameras bieten und ob sich mit ihnen eine präzise Blickverfolgung realisieren lässt wird die Zukunft zeigen.

% ----------------------------------------------------------------------------------------------------------
% Literatur
% ----------------------------------------------------------------------------------------------------------
\renewcommand\refname{Quellenverzeichnis}
\pagebreak
\printbibliography
% ----------------------------------------------------------------------------------------------------------
% Anhang
% ----------------------------------------------------------------------------------------------------------
\clearpage
\pagenumbering{Roman}
\setcounter{page}{1}
\lhead{Anhang \thesection}

\begin{appendix}
\section*{Anhang}
\phantomsection
\addcontentsline{toc}{section}{Anhang}
\addtocontents{toc}{\vspace{-0.5em}}

\section{GUI}
Ein toller Anhang.

\subsection*{Screenshot}
\label{app:screenshot}
Unterkategorie, die nicht im Inhaltsverzeichnis auftaucht.

\end{appendix}


\newpage
\thispagestyle{empty}
\begin{center}
	\vspace*{5em}
	\huge\textbf{Eidesstattliche Erklärung}\\
\end{center}
\vspace{2em}
Ich erkläre an Eides statt, dass ich die vorliegende Arbeit (entsprechend  der genannten Verantwortlichkeit) selbstständig und nur unter Verwendung der angegebenen Quellen und Hilfsmittel angefertigt habe. Die Arbeit wurde bisher in gleicher oder ähnlicher Form weder veröffentlicht noch einer anderen Prüfungsbehörde vorgelegt.    

\vspace{4em}
\begin{minipage}{\linewidth}
	\begin{tabular}{p{15em}p{15em}}
		Datum: &  .......................................................\\
		& \centering (Unterschrift)\\
	\end{tabular}
\end{minipage}

\end{document}
