\subsection{Problemstellung}
Wie in Abschnitt \ref{sec:stand_der_technik} näher erläutert wird, gibt es zwar kommerzielle Anbieter für das Remote Usability Testing, jedoch beschränkt sich deren Angebot entweder auf mobile Websites oder ist mit hohen Kosten verbunden.
Auch die verfügbaren Werkzeuge bieten keine zufriedenstellende Lösung für das entfernte Testen.

Weiterhin besteht zwar die Möglichkeit den Bildschirm von Android Geräten aufzuzeichnen und Interaktionen darzustellen, jedoch wird dafür entweder ein zusätzlicher PC benötigt, oder es besteht keine einfache Möglichkeit diese Aufnahmen auswertbar zu machen und weiterzugeben.

Neben den Bildschirmaufgaben müssen auch Aufgaben und Anweisungen an den Benutzer übermittelt, sowie zusätzliche Metadaten gesammelt werden, um später statistische Aussagen machen zu können.

Für einen Teil dieser Probleme bestehen bereits Lösungsansätze, auf welche im nächsten Abschnitt verwiesen wird.