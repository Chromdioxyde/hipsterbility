\subsection{Bezug der Programmquellen und Dokumentation}
Die Programmquellen der entwickelten, bzw. weiterentwickelten Software ist in dem folgenden \emph{GitHub}-Repository abrufbar:
\begin{center}
	\url{https://github.com/Chromdioxyde/hipsterbility}
\end{center}
Als Versionsverwaltung wird \emph{Git}\footnote{Git Webseite: \url{http://git-scm.com/}} eingesetzt.
Mit dem Befehl
\\
 \texttt{git clone https://github.com/Chromdioxyde/hipsterbility.git} kann eine lokale Kopie erstellt werden.
Alternativ kann der Inhalt des Repositories auf der o.~g. Webseite auch als ZIP-Datei heruntergeladen werden.
Die Quellen der Programme liegen im \emph{IntelliJ IDEA}\footnote{IntelliJ IDEA Website: \url{https://www.jetbrains.com/idea/}} Projektformat vor.
Für die Entwicklung wurde \emph{IntelliJ IDEA Ultimate 13.1.5} verwendet.

\subsection{Motivation}
Bei begrenztem Budget, z.~B. von Studierenden sind umfangreiche Labortests oder das Beauftragen von Dienstleistern oft nicht möglich.

Ein weiterer Faktor ist auch die Zielgruppe für die entwickelte Anwendung.
Bei räumlich verteilten Anwendern sind Labortests durch den logistischen Aufwand nur schwer möglich und auch die finanzielle Belastung steigt durch Transportkosten und evtl. mobiles Equipment.

Deshalb wurde im Wintersemester 2013/2014, zusammen mit Oliver Erxleben, ein Projekt durchgeführt, mit dem Ziel Usability-Tests mit minimalem Aufwand für den Teilnehmer durchzuführen (siehe Abschnitt \ref{subsec:stand_bei_projektbeginn}).

Die damals gesammelten Erkenntnisse und Erfahrungen sollen nun vertieft werden, mit dem Ziel die bestehenden Prototypen zu einem nutzbaren Framework weiter zu entwickeln.


\subsection{Zielsetzung}
Das Ziel ist es, ein Usability-Testing Framework oder Toolkit zu entwickeln, mit dem auch einzelne Entwickler oder kleine Teams ihre Anwendungen und Prototypen mit der Hilfe von Testbenutzern testen können.
Für den eigentlichen Testablauf beim Benutzers soll nur ein Android basiertes Gerät und eine Internetverbindung benötigt werden.
Ziel dabei ist es, das Testen möglichst einfach zu gestalten, ohne Labor und zusätzliches Equipment.

Auf der Seite des Testleiters soll nur ein PC oder Notebook benötigt werden, ohne zusätzliche Hardware.
Die Serverkomponente wurde ausgelagert, für den Fall, dass ein Server zur Verfügung steht, bzw. z.~B. von einer Bildungseinrichtung bereitgestellt wird.