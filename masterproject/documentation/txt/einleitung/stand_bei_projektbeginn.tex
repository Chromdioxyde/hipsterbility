\subsection{Stand bei Projektbeginn}
Dieses Projekt basiert auf einer Arbeit, welche zusammen mit Oliver Erxleben (\href{mailto:oliver.erxleben@hs-osnabrueck.de}{oliver.erxleben@hs-osnabrueck.de}) im Modul \emph{Mensch-Maschine-Kommunikation} (Wintersemester 2013/2014) des Informatik--Masterstudiengangs \emph{Verteilte und mobile Anwendungen} erstellt wurde.
Die Ergebnisse zum Stand der Abgabe können aus dem Git-Repository von Oliver Erxleben unter der folgenden \ac{URL} abgerufen werden:

\url{https://github.com/olivererxleben/hipsterbility/releases/tag/v1.0_semester_final}

Der Vollständigkeit halber wird der Stand der vorausgehenden Arbeit an dieser Stelle kurz wiedergegeben.
Es wurde ein Prototyp erstellt, welcher aus einer Android Bibliothek und einem kleinen \ac{REST}-Server besteht.
Zum Testen der Bibliothek wurde weiterhin eine kleine Android Applikation erstellt.
Insgesamt wurde folgender Funktionsumfang skizziert:
\begin{compactitem}
	\item Android 4.x Bibliothek
	\begin{itemize}
		\item Abrufen von Testsitzungen und Testaufgaben vom Server.
		\item Benutzeroberfläche zum Auswählen und Starten von Testsitzungen.
		\item Aufzeichnen des Kamerabildes der Frontkamera mit Ton, nach dem Starten einer Testsitzung.
		\item Erstellen von Screenshots der Applikation bei Benutzereingaben auf dem Touchscreen und Visualisierung der Eingaben auf dem Screenshot.
		\item Hochladen der gesammelten Daten zum Server.
	\end{itemize}
	\item Node.js basierter Server mit REST-Schnittstelle und Webinterface
	\begin{itemize}
		\item Bereitstellen von Testsitzungen und -aufgaben.
		\item Rudimentäres Benutzermanagement.
		\item Weboberfläche zum Erstellen von Testsitzungen und -aufgaben
		\item Speichern der hochgeladenen Daten in einer Datenbank und im Dateisystem.
		\item Aufbereiten der Screenshots und des Videos von der Frontkamera in ein Ergebnisvideo für die Auswertung.
	\end{itemize}
\end{compactitem}

Die Idee und die ursprüngliche Planung stammen von Oliver Erxleben, welcher auch für die Serverkomponente zuständig war.

Da viele Funktionen in der Implementierung nur skizziert wurden eignet sich der Prototyp nicht für die Verwendung in einer produktiven Umgebung.

Folgende Komponenten wurden aus der vorhergehenden Arbeit übernommen und weiterentwickelt:
\begin{compactitem}
	\item das grundlegende Konzept,
	\item Teile des Datenmodells,
	\item und Quellen der Android Bibliothek und Testapplikation.
\end{compactitem}

Der Server wird auf der Basis von \emph{Java EE 7} neu aufgebaut.
%TODO: Querverweis