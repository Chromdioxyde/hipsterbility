Die Durchführung des Projekts führte zu vertieftem Wissen über verschiedene aktuelle Technologien.
Dazu gehören ein detaillierter Einblick in Java EE und JavaFX, sowie Konzepte wie z.B. Dependency Injection.
Das ursprüngliche Ziel war die Entwicklung eines funktionsfähigen Frameworks, welches aus zwei neu implementierten und einer angepassten Komponente besteht.
Aufgrund der fehlenden Vertrautheit mit den verwendeten Technologien (JavaFX, Java EE 7) und den teilweise schwer ergründbaren Fehlern, die bei der Entwicklung des Servers auftraten, war der Projektumfang deutlich zu groß angesetzt, für eine alleinige Bearbeitung.
Diese Fehleinschätzung führte dazu, dass das Projektziel nicht in vollem Umfang erreicht wurde.

Jedoch wurde eine brauchbare Grundlage mit verschiedenen Beispielen der hier behandelten Konzepte und Technologien geschaffen, auf der sich aufbauen lässt.
Um dieses Framework fertigzustellen, inklusive Unit-Tests und erweiterter Testanalyse müssten jedoch noch viel Zeit investiert werden, was im Rahmen der Projektbearbeitung nicht möglich wäre.
Da das Thema weiterhin spannend ist und Potential enthält wird die Bearbeitung außerhalb des Rahmens des Projekts weitergeführt.
Im nächsten Abschnitt wird kurz auf mögliche Richtungen zur Weiterentwicklung eingegangen.