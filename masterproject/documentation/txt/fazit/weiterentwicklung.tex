\subsection{Weiterentwicklung}
\subsubsection{Integrierte Blickverfolgung}
Projekte wie \emph{Opengazer} \cite{TheInferenceGroup..2009} zeigen eindrucksvoll, dass Eye-Tracking bzw. Gaze-Tracking auch ohne spezielles Equipment möglich ist.
Mit Hilfe einer einfachen Webcam gelang dem Entwicklerteam das Verfolgen von Augenbewegungen und die Visualisierung des Fokuspunktes auf dem Computerdisplay.
Die quelloffene Software benötigt für die korrekte Funktion das Markieren von Referenzpunkten im Gesicht auf dem Kamerabild, sowie eine Kalibrierung des Blickes mit Referenzpunkten auf dem Display.

Ein schwerwiegender Nachteil der aktuellen Version ist allerdings, dass der Kopf mög\-lich\-st still gehalten werden und die Webcam fixiert werden muss.
Selbst wenn die Software auf mobile Betriebssysteme portiert und ausgeführt werden könnte, wäre es nur schwer möglich die zuvor genannten Bedingungen zu erfüllen.
Das Testgerät müsste fixiert werden und auch die Testperson dürfte sich beim Test möglichst wenig bewegen.
Die Qualität der Messungen wäre auch bei perfekten Bedingungen fraglich, da schon kleine Messungenauigkeiten den Wert der Messpunkte stark reduzieren würden, da die Displays von mobilen Geräten sehr viel kleiner sind als bei Desktop-PCs oder Notebooks.

\subsubsection{Blickerkennung mit zusätzlicher Hardware}
Eine alternative zur Blickverfolgung mit der integrierten Kamera von mobilen Geräten könnte \emph{The Eye Tribe} \cite{TheEyeTribe.2014} bieten.
Diese bieten einen kompakten Eye-Tracker für \$99 an.
Die Hardware ist mit 1,9 x 20 x 1,9 cm Ausmaßen sehr kompakt und mit 70 g auch leicht genug für den mobilen Einsatz, zumindest an Tablet"~PCs.
Aktuell ist die Software nur unter Microsoft Windows lauffähig, eine Android Version befindet sich, laut Herstellerwebsite, in der Entwicklung.

Blickverfolgung auf einem Smartphone ohne externe Hardware könnte schon in naher Zukunft möglich sein.
Durch die Verwendung von mehreren Kameras in der Gerätefront kann das \emph{Amazon Fire Phone} \cite{amazon.de.2014} schon jetzt die Perspektive des Bildschirminhalts dynamisch an den Betrachtungswinkel des Benutzers anpassen.
Zwar wird das Eye-Tracking vom aktuellen \emph{Fire Phone \ac{SDK}} (noch) nicht unterstützt, lediglich das Head-Tracking, jedoch ist das Smartphone erst seit kurzem in den U.S.A. verfügbar und der Verkaufsstart in Deutschland steht noch aus (stand September 2014).
Welche Möglichkeiten die zusätzlichen Kameras bieten und ob sich mit ihnen eine präzise Blickverfolgung realisieren lässt wird die Zukunft zeigen.
