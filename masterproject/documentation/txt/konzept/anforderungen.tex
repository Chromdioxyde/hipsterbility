\subsection{Remote Usability Testing}\label{subsec:remote_usability_testing}
Remote Usability Testing kann in zwei Varianten durchgeführt werden, moderiert oder nicht moderiert.
Bei der nicht moderierten Variante haben Studienteilnehmer bzw. Testbenutzer nur die ihnen gegebene Aufgabenstellung und Informationen zur Verfügung.
Diese Testmethode ist beispielsweise beim Testen von Webseiten gängig.
Mittlerweile gehört die Untersuchung von mobilen Webseiten zu dem Angebot einiger Dienstleister (siehe Abschnitt \ref{subsec:usability_testing_anbieter}).

Um eine solche Studie durchzuführen werden üblicherweise die folgenden vier Schritte benötigt \cite[Vgl.][]{Sourcy.2010}:
\begin{compactdesc}
	\item[Definieren der Studie:] 
	Zusammenstellen von Aufgaben und Fragen, welche von den Teilnehmern bearbeitet werden sollen.
	\item[Rekrutieren von Teilnehmern:] 
	Testpersonen werden entweder direkt oder durch Dienstleister kontaktiert und gebeten an der Studie teilzunehmen.
	\item[Verschicken von Einladungen und Starten des Tests:] 
	Wenn die zuvor genannten Schritte erfüllt sind, kann die Studie gestartet werden und die Teilnehmer werden z.~B. per E-Mail benachrichtigt.
	\item[Analysieren der Ergebnisse:] Das Auswerten der Testergebnisse wird häufig von Werkzeugen unterstützt, die von Dienstleistern bereitgestellt werden.
\end{compactdesc}
Die Dauer der Tests sollen drei bis 15 Minuten umfassen und es sollten nicht mehr als drei bis fünf Testaufgaben gestellt werden, da eine zu hohe Dauer und zu viele Aufgaben die Erfolgsrate bei der Erfüllung der Aufgaben reduzieren und für eine höhere Ausstiegsrate sorgen.\cite[Vgl.][]{Sourcy.2010}

Diese Methode soll hier durch Software zum nativen Testen von Android Applikationen nachgebildet werden.

Der nächste Abschnitt stellt die geplante Architektur der Anwendung dar.