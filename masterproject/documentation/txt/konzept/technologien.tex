\subsection{Evaluierung und Auswahl von Technologien}
Im Gegensatz zu dem Ursprünglichen Projekt (siehe Abschnitt \ref{subsec:stand_bei_projektbeginn}) liegt hier nicht der Fokus auf der möglichst schnellen und einfachen Entwicklung von Prototypen, sondern eines weitgehend einsatzfähigen Frameworks.

Aus diesem Grund werden Technologien unter verschiedenen Gesichtspunkten neu evaluiert und ausgewählt.

Da das Framework später quelloffen und kostenlos bereitgestellt werden soll, wird der Einsatz kommerzieller Software und Bibliotheken von Anfang an ausgeschlossen.

Nachfolgend werden die einzelnen Komponenten des Frameworks näher betrachtet und mögliche Technologien, sowie die finale Auswahl, dargestellt.

Besonderes Gewicht bei der Auswahl bekommt die Zielgruppe, welche aus Android-Entwicklern besteht.
Android Applikationen werden vorwiegend in der Java-Pro\-gram\-mier\-spra\-che entwickelt, weshalb zumindest deren Grundkenntnisse vorausgesetzt werden können.
Aus den Android \ac{SDK} Systemvoraussetzungen \cite[vgl.][]{AndroidOpenSourceProject.2014c} geht außerdem hervor, dass für das Entwickeln ein Computer vorhanden sein muss, welcher als Betriebssystem ein halbwegs aktuelles Microsoft Windows, Max OS X oder Linux besitzt, auf dem mindestens das \ac{JDK} 6 vorhanden ist.

Begonnen wird mit der Auswahl der Server-Plattform, da diese die Auswahl der Technologie für den Testleiter-Client beeinflusst.
Besonderer Wert wird auch auf die schnelle und einfache Entwicklung gelegt.

\subsubsection{Server}
Da das Framework ein verteiltes System mit klassischer Client"~Server Architektur ist (siehe Abschnitt \ref{subsec:architektur}), wird eine Serverkomponente benötigt.

Aus den allgemeinen Anforderungen werden nun Kriterien zur Auswahl der Server-Plattform definiert.

Anforderungen an den Server:
\begin{compactitem}
	\item Implementierung einer \ac{REST}-\ac{API},
	\item Anbindung eines Datenbanksystems und
	\item Unterstützung der Java Programmiersprache für geteilte Klassen.
\end{compactitem}

Für den Prototyp wurde \emph{Node.js} (\url{http://nodejs.org/}) eingesetzt.
\emph{Node.js} ist ein ereignisgesteuertes Framework, für welches vorwiegend in der JavaScript-Sprache \cite[vgl.][]{Joyent.2014} programmiert wird.

Zwar existiert ein Bridge-\ac{API} um Java Code in JavaScript verwenden zu können .
Die Aufrufe erfolgen intern über \ac{JNI} und bieten somit keinen Ersatz für eine vollständige Java (EE) Umgebung. \cite[Vgl.][]{Ferner.2014}

Das geteilte Verwenden von Java-Klassen in Client und Server wäre demnach nur eingeschränkt möglich.

Eine Alternative, welche die o.g. Anforderungen erfüllt, sind Java EE Application Server.
Sie bieten den vollen Java Sprachumfang und erlauben das Teilen von Klassen zwischen verschiedenen Modulen.
Die aktuelle Java EE 7 Spezifikation beinhaltet u.~a. \emph{Java Persistence 2.1} für den Datenbankzugriff und die \emph{Java API for RESTful Web Services (JAX-RS) 2.0} zum Aufbau eines RESTful Webservice.
\cite[Vgl.][]{Oracle.2014}

Als Java EE 7 Implementierung wird der \emph{Glassfish 4.1} (\url{https://glassfish.java.net/})
\ac{AS} verwendet, da er die Referenzimplementierung der Plattform darstellt \cite[vgl.][14]{OracleCorporation.2013b}.
Außerdem ist er quelloffen (Open Source Edition) und kostenlos verfügbar.

Basierend auf dieser Auswahl erfolgt im nächsten Abschnitt die Technologie- und Plattformauswahl für den Testleiter-Client. 

\subsubsection{Testleiter-Client}
Wie auch beim Server soll der Client nicht von kostenpflichtiger Software oder Bibliotheken abhängen.
Es bietet sich an, auch beim Client die Java Programmiersprache als Basis zu nehmen.
Der Client ist eine Anwendung mit grafischer Oberfläche.
Zum Erstellen einer solchen stehen verschiedenen Standardbibliotheken bereit.
Einerseits wäre ein Web-Client denkbar, der auf der Servlet-Engine und \ac{JSP} bzw. \ac{JSF}/Facelets basiert.
Eine Neuerung in Java EE 7 ist die \ac{HTML} 5 Unterstützung \cite[vgl.][5]{OracleCorporation.2013b}, mit welcher auch die WebSocket Technologie eingeführt wurde.
Sie erlaubt eine bidirektionale, ereignisgesteuerte Kommunikation mit niedriger Latenz zwischen Client und Server.

Wie im Konzept beschrieben ist es nötig zwei oder mehr dynamische Inhalte in einer Benutzeroberfläche anzuzeigen und zeitlich zu synchronisieren.
Es wurde keine einfache Möglichkeit gefunden dies mit den zuvor genannten Technologien zu erreichen.
Es gibt zwar mehrere HTML5 Beispielimplementierungen, die das synchrone Abspielen von zwei oder mehr Videos zeigen sollen \cite{Waldron.2011,HTML5demos.2011}, dabei wurde jedoch jeweils das selbe Video mit der gleichen Bildwiederholrate, Auflösung, Dauer usw. gewählt.
Diese Demonstration ist nicht ausreichend um die Entwicklung des Clients zu bestimmen.

Aus diesem Grund gibt es JavaScript Frameworks wie \emph{Popcorn.js} (\url{http://popcornjs.org/}), die es erlauben z.~B. Videos und andere Medien/Inhalte synchron darzustellen.
Diese Framework bietet zwar die nötige Funktionalität, erfordert jedoch zusätzlich den Einsatz der JavaScript Sprache und in Kombination mit WebSockets um das gewünschte Ergebnis zu erreichen. \cite[Vgl.][]{bocoup.2014}

Um keine Brüche in der Programmiersprache zu haben und um die bekannte Umgebung von Java/Android Entwicklern nicht zu weit zu verlassen wurde nach einer Alternative auf Basis von Java SE gesucht.
Das Swing Oberfächenframework bekommt mit der Einführung von Java 8 Konkurrenz durch JavaFX, welches zwar grundsätzlich schon seit 2007 verfügbar ist, jedoch bisher nur separat vertrieben wurde. 
Im Vergleich zum angestaubten Swing Framework bietet JavaFX u.~a. moderne Oberflächen auf \ac{XML}-Basis, eine umfangreiches Medienframework und hardwarebeschleunigte 3D Grafik.
Seit dem 18.03.2014 ist es zudem Bestandteil des \ac{JDK} 8 und auch in Java SE 8 enthalten \cite[Vgl.][xxiv-xxvi]{Dea.2014}.
Besonders aufgrund der neuen und umfangreichen Medienfunktionen wird beim Testleiter Client auf JavaFX~8 gesetzt.

Zunächst wird jedoch im folgenden Abschnitt beschrieben, wie der Server realisiert wird, da dieser das Grundgerüst des Frameworks darstellt. 