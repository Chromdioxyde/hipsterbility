\subsection{RESTFul Webservice}
Um die Android Bibliothek und den Testleiter-Client anzubinden wird eine \ac{REST}-\ac{API} verwendet.
Da der \emph{Glassfish} \ac{AS} die Grundlage für die Serverkomponente darstellt, kommt die \ac{JAX-RS} 2.0 Implementierung \emph{Jersey} (\url{https://jersey.java.net/}) zum Einsatz, welche beim \ac{AS} mitgeliefert wird.
Als Referenz für die Implementierung der Schnittstelle und den Aufbau der Ressourcen dient \citetitle{Burke.2014} von \citeauthor{Burke.2014} \cite{Burke.2014}.

Eine möglichst lose Kopplung der verschiedenen Klassen wird durch die Verwendung von Interfaces und Dependency Injection erreicht.
Ein gutes Beispiel dafür zeigt Robert Leggett mit seinem GitHub Projekt \emph{jersey\_restful\_webservice}\footnote{GitHub Repository: \url{https://github.com/Robert-Leggett/jersey_restful_webservice}}, welches vom Ressourcenaufbau eher dem \ac{REST}-\ac{RPC} Schema folgt.

Aktuell wird nur das \ac{JSON} Datenformat unterstützt.
Für das (Un--)Marshalling wird serverseitig die \emph{Jackson} (\url{https://github.com/FasterXML/jackson}) Bibliothek verwendet.

Eine Implementierung des \ac{HATEOAS} Prinzips \cite[vgl.][11-13]{Burke.2014} wurde nicht vorgenommen, da die entwickelte \ac{API} nicht öffentlich zugänglich sein wird.
Außerdem wird sie aktuell nur von den eigenen Clients genutzt, welchen der Aufbau der \ac{API} nd Ressourcen bekannt ist.
Ein detaillierte Beschreibung der Ressourcen mit Pfaden und \ac{HTTP}"~Methoden befindet sich in Abschnitt \ref{subsubsec:rest_ressourcen}.

\subsubsection{Java REST-API mit JAX-RS Annotationen und CDI}

\subsubsection{Implementierte Ressourcen\label{subsubsec:rest_ressourcen}}
Die Ressourcen werden als Gruppen von Objekten angesehen und es werden Nomen im Plural verwendet \cite[vgl.][20\psqq]{Burke.2014} um die Benennung einheitlich zu gestalten (z.B. \texttt{users, sessions, devices}).
Dies weicht vom Datenbankmodell ab, da dort die Tabellen im Singular benannt sind (siehe Abschnitt \ref{subsec:datenbank}).

Wie auch bei den \acp{DAO} in der Persistenzschicht der Serveranwendung wurden die vier grundlegenden Datenoperationen \ac{CRUD} implementiert.
Die Zuordnung zwischen Operation und \ac{HTTP}"~Methoden bzw. \ac{SQL}"~Befehlen ist in Tabelle \ref{tbl:crud} zu sehen.

\begin{minipage}[t]{\textwidth}
	\centering
	\begin{tabu}{|X|>{\ttfamily}X|>{\ttfamily}X|}
	\rowfont[l]{\normalfont\bfseries} 
		\hline Operation & HTTP & SQL \\ 
		\hline Create & POST & INSERT \\ 
		\hline Read / Retrieve & GET & SELECT \\ 
		\hline Update / Modify & PUT & UPDATE \\ 
		\hline Delete & DELETE & DELETE \\ 
		\hline 
	\end{tabu}
	\captionof{table}{Zuweisung der CRUD Operationen und HTTP-Methoden und SQL-Befehlen.}
	\label{tbl:crud}
\end{minipage}

\subsubsection{Aufbau von Ressourcen und Zugriffsrechte}
Die Ressourcen sind flach gehalten.
Je nach Benutzerrolle führen Aufrufe der Methoden zu verschiedenen Ergebnissen.
Bei Ressourcen mit Benutzerbezug (z.~B. Geräte und Testsitzungen) bekommt ein angemeldeter Benutzer in der Rolle \texttt{USER} nur Objekte, welche mit ihm in Relation stehen.
Dadurch wird sichergestellt, dass einfache Benutzer nicht auf Daten anderer Benutzer zugreifen können.
Benutzer in der Rolle \texttt{ADMIN} können hingegen auf alle Objekte in einer Ressource zugreifen.

Das erstellen von Objekten mit Benutzerbezug über die \texttt{POST}"~Methode erfordert in der Rolle \texttt{ADMIN} zusätzlich zu dem zu erstellenden Objekt auch noch die \texttt{ID} des Benutzers, für welchen diese erstellt werden soll, sofern die Objekte keine bidirektionale Beziehung haben und sich der Bezug daraus herstellen lässt. 

Nachfolgend werden die Resourcen mit den implementierten \ac{HTTP}"~Methoden aufgelistet.
Die Pfadangabe erfolgt relativ zur Basis"~\ac{URL}.
Angaben in geschweiften Klammern sind Pfad-Parameter, welchen im Aufruf ein Wert zugewiesen ist.
%TODO: fehlende Ressourcen nachpflegen


\begin{minipage}[t]{\textwidth}
	\centering
	\begin{tabu}{>{\ttfamily}X[2]X[1]>{\ttfamily}X[3]>{\small}X[4]}
		\rowfont[l]{\normalfont\bfseries\normalsize}
		Rollen & Methode & Pfad &   Beschreibung\\
		USER & GET & /users &   Daten des angemeldeten Benutzers\\ 
		ADMIN & GET & /users &  Liste aller Benutzer\\
		USER & PUT & /users & Bestätigung oder Fehlermeldung\\
		ADMIN & PUT & /users/\{id\} & Bestätigung oder Fehlermeldung\\ 
		ALL, ADMIN & POST & /users &  ID des neu erstellten Benutzers\\
		ADMIN & GET & /users/\{id\} & Ausgabe des Benutzers mit der ID \texttt{id}\\
		ADMIN & DELETE & /users/\{id\} & Löschen des Benutzers mit der ID \texttt{id}\\
		\\
		USER & GET & /devices & Liste mit Geräten des Benutzers\\
		ADMIN & GET & /devices & Liste mit allen Geräten\\
		ADMIN & GET & /devices/\{id\} & Gerät mit der ID \texttt{id}\\
		
		ADMIN & DELETE & /devices/\{id\} & Gerät mit der ID \texttt{id}\\
		ADMIN & PUT & /devices/\{id\} & Gerät mit der ID \texttt{id}\\
	\end{tabu}
	\captionof{table}{REST"~Ressourcen mit HTTP"~Methoden, Pfad, Benutzerrolle und Beschreibung.}
	\label{tbl:rest_ressourcen}
\end{minipage}


