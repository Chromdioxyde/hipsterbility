\subsection{RESTFul Webservice}
Um die Android Bibliothek und den Testleiter-Client anzubinden wird eine \ac{REST}-\ac{API} verwendet.
Da der \emph{Glassfish} \ac{AS} die Grundlage für die Serverkomponente darstellt, kommt die \ac{JAX-RS} 2.0 Implementierung \emph{Jersey} (\url{https://jersey.java.net/}) zum Einsatz, welche beim \ac{AS} mitgeliefert wird.
Als Referenz für die Implementierung der Schnittstelle und den Aufbau der Ressourcen dient \citetitle{Burke.2014} von \citeauthor{Burke.2014} \cite{Burke.2014}.

Eine möglichst lose Kopplung der verschiedenen Klassen wird durch die Verwendung von Interfaces und Dependency Injection erreicht.
Ein gutes Beispiel dafür zeigt Robert Leggett mit seinem GitHub Projekt \emph{jersey\_restful\_webservice}\footnote{GitHub Repository: \url{https://github.com/Robert-Leggett/jersey_restful_webservice}}, welches vom Ressourcenaufbau eher dem \ac{REST}-\ac{RPC} Schema folgt.

Aktuell wird nur das \ac{JSON} Datenformat unterstützt.
Für das (Un--)Marshalling wird serverseitig die \emph{Jackson} (\url{https://github.com/FasterXML/jackson}) Bibliothek verwendet.

Eine Implementierung des \ac{HATEOAS} Prinzips \cite[vgl.][11-13]{Burke.2014} wurde nicht vorgenommen, da die entwickelte \ac{API} nicht öffentlich zugänglich sein wird.
Außerdem wird sie aktuell nur von den eigenen Clients genutzt, welchen der Aufbau der \ac{API} nd Ressourcen bekannt ist.
Ein detaillierte Beschreibung der Ressourcen mit Pfaden und \ac{HTTP}"~Methoden befindet sich in Abschnitt \ref{subsubsec:rest_ressourcen}.

\subsubsection{Java REST-API mit JAX-RS Annotationen und CDI}
Die Serveranwendung besteht insgesamt aus den folgenden drei Schichten:
\begin{compactenum}
	\item Data Access Layer
	\item Service Layer
	\item Resource Layer
\end{compactenum}
Der Data Access Layer abstrahiert die Datenbankanbindung und implementiert die \ac{CRUD}-Methoden für den Datenzugriff.
Für jeden Entitätentyp wurde ein eigenes \ac{DAO} erstellt, lediglich die Verwaltung von \texttt{FileEntity} Subklassen wurde in einem \ac{DAO} zusammengefasst, da sich diese Objekte in ihrer Struktur sehr ähnlich sind.
Der Service Layer nutzt ein oder mehrere \acp{DAO} für die Persistenz und abstrahiert diese wiederum für den Resource Layer, welcher die \ac{REST}-\ac{API} bildet und selbst auch ein oder mehrere Services nutzt.

Die \ac{DAO} und Service~Objekte werden per \ac{CDI} in die Objekte der Jeweils nächsten Schicht injiziert.
Dies eliminiert einen Teil der Objektverwaltung zur Laufzeit, da sich der Container derer annimmt.
Der Container scannt die Klassen automatisch nach entsprechenden Annotationen ab, sofern die Bindings nicht manuell spezifiziert werden.
Neben fast allen Java-Klassen lassen sich u.~a. auch Session Beans, Java EE Ressourcen und Kontext Objekte injizieren.
Außerdem besteht so die Möglichkeit verschiedene Implementierungen für ein Interface bereitzustellen, von denen eine während der Laufzeit ausgewählt wird. \cite[Vgl.][497\psqq]{Juneau.2013}

Bei den \ac{DAO} und Services wurde die \texttt{@Singleton} Annotation gewählt, wodurch diese als Singleton Session Beans deklariert werden \cite[vgl.][]{Oracle.2013b}.
Dies wurde gewählt, weil die Objekte keinen Zustand speichern -- \texttt{@Stateless} hätte dies auch erfüllt -- und der Dienst insgesamt nicht auf die Bewältigung von einer Großen Anzahl von Anfragen ausgelegt ist.


\subsubsection{Implementierte Ressourcen\label{subsubsec:rest_ressourcen}}
Die Ressourcen werden als Gruppen von Objekten angesehen und es werden Nomen im Plural verwendet \cite[vgl.][20\psqq]{Burke.2014} um die Benennung einheitlich zu gestalten (z.B. \texttt{users, sessions, devices}).
Dies weicht vom Datenbankmodell ab, da dort die Tabellen im Singular benannt sind (siehe Abschnitt \ref{subsec:datenbank}).

Wie auch bei den \acp{DAO} in der Persistenzschicht der Serveranwendung wurden die vier grundlegenden Datenoperationen \ac{CRUD} implementiert.
Die Zuordnung zwischen Operation und \ac{HTTP}"~Methoden bzw. \ac{SQL}"~Befehlen ist in Tabelle \ref{tbl:crud} zu sehen.

\begin{minipage}[t]{\textwidth}
	\centering
	\begin{tabu}{|X|>{\ttfamily}X|>{\ttfamily}X|}
	\rowfont[l]{\normalfont\bfseries} 
		\hline Operation & HTTP & SQL \\ 
		\hline Create & POST & INSERT \\ 
		\hline Read / Retrieve & GET & SELECT \\ 
		\hline Update / Modify & PUT & UPDATE \\ 
		\hline Delete & DELETE & DELETE \\ 
		\hline 
	\end{tabu}
	\captionof{table}{Zuweisung der CRUD Operationen und HTTP-Methoden und SQL-Befehlen.}
	\label{tbl:crud}
\end{minipage}

\subsubsection{Aufbau von Ressourcen und Zugriffsrechte}
Die Ressourcen sind flach gehalten.
Je nach Benutzerrolle führen Aufrufe der Methoden zu verschiedenen Ergebnissen.
Bei Ressourcen mit Benutzerbezug (z.~B. Geräte und Testsitzungen) bekommt ein angemeldeter Benutzer in der Rolle \texttt{USER} nur Objekte, welche mit ihm in Relation stehen.
Dadurch wird sichergestellt, dass einfache Benutzer nicht auf Daten anderer Benutzer zugreifen können.
Benutzer in der Rolle \texttt{ADMIN} können hingegen auf alle Objekte in einer Ressource zugreifen.

Das erstellen von Objekten mit Benutzerbezug über die \texttt{POST}"~Methode erfordert in der Rolle \texttt{ADMIN} zusätzlich zu dem zu erstellenden Objekt auch noch die \texttt{ID} des Benutzers, für welchen diese erstellt werden soll, sofern die Objekte keine bidirektionale Beziehung haben und sich der Bezug daraus herstellen lässt. 

Nachfolgend werden die Resourcen mit den implementierten \ac{HTTP}"~Methoden aufgelistet.
Die Pfadangabe erfolgt relativ zur Basis"~\ac{URL}.
Angaben in geschweiften Klammern sind Pfad-Parameter, welchen im Aufruf ein Wert zugewiesen ist.

\begin{minipage}[t]{\textwidth}
	\centering
	\begin{tabu}{>{\ttfamily}X[2]X[1]>{\ttfamily}X[3]>{\small}X[4]}
		\rowfont[l]{\normalfont\bfseries\normalsize}
		Rollen & Methode & Pfad &   Beschreibung\\
		USER & GET & /users &   Abrufen der Daten des angemeldeten Benutzers\\ 
		ADMIN & GET & /users &  Liste aller Benutzer\\
		USER & PUT & /users & Angemeldeten Benutzer aktualisieren\\
		ADMIN & PUT & /users/\{id\} & Benutzer mit der ID \texttt{id} aktualisieren\\ 
		ALL, ADMIN & POST & /users &  Neuen Benutzer erstellen\\
		ADMIN & GET & /users/\{id\} & Ausgabe des Benutzers mit der ID \texttt{id}\\
		ADMIN & DELETE & /users/\{id\} & Löschen des Benutzers mit der ID \texttt{id}\\
		\\
		USER & GET & /devices & Liste mit Geräten des Benutzers\\
		ADMIN & GET & /devices & Liste mit allen Geräten\\
		ADMIN & GET & /devices/\{id\} & Abrufen Gerät mit der ID \texttt{id}\\
		
		ADMIN & DELETE & /devices/\{id\} & Löschen des Geräts mit der ID \texttt{id}\\
		ADMIN & PUT & /devices/\{id\} & Aktualisieren des Geräts mit der ID \texttt{id}\\
	\end{tabu}
	\captionof{table}{Einige REST"~Ressourcen mit HTTP"~Methoden, Pfad, Benutzerrolle und Beschreibung.}
	\label{tbl:rest_ressourcen}
\end{minipage}

Tabelle \ref{tbl:rest_ressourcen} zeigt den Aufbau einiger Ressourcen mit zugelassenen Benutzerrollen, Pfaden und Parametern.
Diese Ressourcen sollen als Beispiel dienen, für die verschiedenen Ebenen des Ressourcenaufbaus, denn je nach Benutzerrolle verhalten sich bestimmte Ressourcen anders als bei anderen Rollen.

Für Ressourcen die Objekte mit Objektlisten als Klassenvariablen besitzen wurde eine mehrstufige Hierarchie aufgebaut, nach dem Schema \texttt{/resource/\{id\}/subresource/\{id\}}.

Wie diese Ressourcen und die Anmeldedaten der Benutzer gesichert werden, ist im nächsten Abschnitt beschrieben.
