\subsection{Realisierung des Clients für Testleiter und Administratoren}
Der Testleiter Client wurde auf Basis von JavaFX mit dem \ac{JDK} 8 erstellt.
JavaFX ist seit Java 8 Bestandteil der Java SE Laufzeitumgebung und unterstützt viele Neuerungen wie Lambda-Methoden und Property-Bindings.
Neben der einfachen \ac{GUI} Entwicklung, wird auch das Rendern von dreidimensionalen Objekten und Szenen mit Hardwarebeschleunigung unterstützt.
Bei JavaFX bietet außerdem Zahlreiche Methoden an um Animationen zu gestalten und Medien wiederzugeben und zu manipulieren.

Das Hauptnachschlagewerk für die Implementierung wird das Buch \citetitle{Dea.2014} von \citeauthor{Dea.2014} \cite{Dea.2014}.

Nachfolgend wir die Erstellung von grafischen Oberflächen in JavaFX kurz beschrieben.

\subsubsection{Grafische Oberflächen mit dem JavaFX Scene Builder 2.0}
Ähnlich der \ac{XML} Layouts bei der Android Plattform, bietet JavaFX die Möglichkeit Oberflächen deklarativ und außerhalb des Quellcodes zu definieren.
Dies geschieht durch \ac{XML} basierte \texttt{FXML}-Dateien.
Um diese nicht manuell erstellen zu müssen, stellt Oracle das Werkzeug \emph{Scene Builder} \cite{OracleCorporation.2014c} zur Verfügung, mit welchen sich Oberflächen im \ac{WYSIWYG} Stil erstellen lassen.

Abbildung \ref{fig:scenebuilder_screenshot} zeigt, den \emph{Scene Builder} mit einer geöffneten \texttt{FXML}-Datei.
Wichtige Elemente für das Erstellen von Oberflächen sind die Zuweisung einer Controller Klasse, die Vergabe der \texttt{fx:id} und bei Schaltflächen oder interaktiven \ac{GUI}"~Elementen die \texttt{On Action} Methode.

Die Möglichkeit eine Controller-Klasse zu vergeben zeigt, dass die Verwendung des \ac{MVC} oder \ac{MVP} Entwurfsmuster vorgesehen, bzw. leicht möglich, ist.  
Bei der Implementierung wurde das \ac{MVP} Entwurfsmuster umgesetzt, welches sich dadurch auszeichnet, dass möglichst wenig Programmlogik im View, der Oberfläche, verwendet wir.
Der Presenter beherbergt einen Großteil der Programmlogik und stellt das Bindeglied zwischen Model und View dar. \cite[Vgl.][74]{Dea.2014}

Wenn ein \ac{GUI}-Element eine zugewiesene \texttt{fx:id} (Abbildung \ref{fig:scenebuilder_screenshot} rechts oben) besitzt, so kann es mit der \texttt{@FXML} Annotation an einer Objektvariable in der zugewiesenen Controller-Klasse direkt injiziert werden.
Dies erleichtert die Entwicklung, da diese Objekte nicht mehr manuell instanziert und der Oberfläche zugewiesen werden müssen.
Das injizieren mit der \texttt{@FXML} Annotation ist auch bei Methoden möglich, denen eine \texttt{On Action} Methode vergeben wird.
Dadurch entfällt das manuelle Zuweisen eines \texttt{ActionListeners}.

\subsubsection{Properties und Binding}
Das neu eingeführte Property-Binding ermöglicht das Verknüpfen von \ac{GUI}-Elementen, teilweise auch als \emph{Node}, \emph{Control} oder \emph{View} bezeichnet, mit Properties, welche in der einfachsten Form Wrapper für primitive Objekte darstellen.

Nachfolgend einige Beispiele für Property-Klassen (\texttt{javafx.beans.property} ) in JavaFX:
\begin{compactitem}
	\item \texttt{SimpleBooleanProperty}
	\item \texttt{ReadOnlyBooleanWrapper}
	\item \texttt{SimpleIntegerProperty}
	\item \texttt{ReadOnlyIntegerWrapper}
	\item \texttt{SimpleStringProperty}
	\item \texttt{ReadOnlyStringWrapper}
\end{compactitem}
Diese Beispiele zeigen einige der Property Klassen, welche in zwei Gruppen eingeteilt werden können, Properties mit Lese-/Schreibzugriff und schreibgeschützte Properties.
Der Hauptvorteil dieser Wrapper-Klassen liegt darin, dass sie sowohl die Schnittstelle \texttt{Obervable}, als auch \texttt{ObservableValue} implementieren und sich mit \texttt{addListener()} \texttt{ChangeListener} und \texttt{InvalidationListener} registrieren lassen.
Dadurch müssen die Werte nicht durch Polling abgefragt werden und die Listener werden bei Ereignissen benachrichtigt.
Die Implementierung der Listener ist durch Java-Klassen, anonyme Klassen oder Lambda-Ausdrücke möglich. \cite[Vgl.][75-79]{Dea.2014}

Ein weitere Eigenschaft von Properties sind Bindings.
Dadurch lassen sich zwei oder mehrere Werte synchron halten, indem Properties an \ac{GUI} Elemente gebunden werden.
Einfache Bindings, bei denen nur die Werte synchronisiert werden, gibt es in zwei Varianten, unidirektional und bidirektional. 
Beim unidirektionalen Binding mit \texttt{bind()} wird die Property aktualisiert, an welcher \texttt{bind()} aufgerufen wurde.
Bei bidirektionaler Wertebindung mit \texttt{bindBidirectional()} werden die Werte in beide Richtungen synchronisiert.
Neben der einfachen Wertebindung gibt es noch weitere Bindings die sich wie boolsche Operatoren oder mathematische Funktionen verhalten. \cite[Vgl.][79-81]{Dea.2014}

Diese werden hier jedoch nicht weiter behandelt. 

\subsubsection{Dependency Injection mit afterburner.fx}
\emph{afterburner.fx} \cite{Bien.2014} ist ein minimalistisches Framework, bestehend aus zwei Java Klassen, welches die beiden Prinzipien \emph{Convention over Configuration} und \emph{Dependency Injection} in JavaFX Anwendungen ermöglicht.
Durch das Befolgen eines vorgegebenen Dateischemas und Packageaufbaus können Views praktisch ohne Code erstellt werden, einzig durch eine \texttt{FXML}"~Datei und eine Klasse die von \texttt{FXMLView} erbt.

Das Vorgegebene Schema sieht für \enquote{MyClass} etwa wie folgt aus:

\dirtree{%
.1 /.
.2 MyClassPresenter.java.
.2 MyClassView.java.
.2 myclass.css.
.2 myclass.properties.
.2 myclass.fxml.
}

Es werden eine \texttt{*View.java} und eine \texttt{*Presenter.java} Klasse erwartet, sowie eine \ac{CSS}"~Datei, für die Gestaltung der View, eine \texttt{*.properties}-Datei mit Key-Value Inhalt, z.~B. für Strings, und die \texttt{FXML}"~Datei für das \ac{GUI}"~Layout erwartet.
Die \texttt{.css} und \texttt{.properties} Dateien benötigen nicht zwingend einen Inhalt, sollten jedoch vorhanden sein.

Neben der vereinfachten Oberflächenerstellung bietet das Framework noch eine zweite nützliche Eigenschaft.
Es erlaubt, ohne weitere Konfiguration, Java EE Annotationen (\texttt{javax.inject.*}) für Dependency Injection zu benutzen, wozu andernfalls Frameworks wie \emph{Google Guice}\footnote{Google Guice Webseite: \url{https://code.google.com/p/google-guice/}}, benötigt würden, die eine umfangreiche Konfiguration verlangen. \cite[Vgl.][]{Bien.2014}

Die Verbindung zum Server wird auch hier über einen \ac{REST}"~Client hergestellt.

\subsubsection{REST-Client}
Als \ac{REST}-Client wird der Client der Jersey 2 Implementierung verwendet.
Für Aufrufe erfolgen fließend und erlauben es unnötige Zuweisungen zu vermeiden, z.~B.
\begin{center}
\texttt{Response r = client.target(getBaseUri()).path(<Pfad>).request().head();}
\end{center}

Aus dem Response Objekt können Objekte mit \texttt{.readEntity()} abgerufen werden, dazu muss der Typ des Abzurufenden Objekts im Aufruf übergeben werden und die Entitäten werden automatisch deserialisert. \cite[Vgl.][113\psqq]{Burke.2014}


