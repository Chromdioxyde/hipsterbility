\subsection{Testen des Servers}
Das Testen des Servers ist aufgrund der Dependency Injection nicht einfach mit JUnit testbar, da JUnit keinen Servlet Container bereitstellt.
Aus diesem Grund wurden während der Entwicklung verschiedenen Integrationstests durchgeführt, teilweise auch mit JUnit.
Auf die Erstellung einer vollständigen Testsuite der Komponenten wurde aus Zeigründen verzichtet.
Das dies ein Fehler war zeigte sich im Verlauf der Entwicklung, da z.~B. führten Veränderungen an Entitätsklassen zu Fehlermeldungen, dass die Entity dem \ac{JPA} \texttt{EntityManager} nicht bekannt sei.
Wie sich später herausstellte wurde durch das erneute Deployment der Anwendung die \texttt{EntityManagerFactory} im Glassfish \ac{AS} nicht geschlossen und neu erstellt, sondern die vorherige Version weiterverwendet, was zu schwer erklärbaren Fehlermeldungen führte.

Die \ac{REST}-\ac{API} wurde Anfangs mit der Browsererweiterung \emph{Postman}\footnote{Postman im Chrome Webstore: \url{https://chrome.google.com/webstore/detail/postman-rest-client/fdmmgilgnpjigdojojpjoooidkmcomcm}} für den Google Chrome Browser, später mit Unit-Tests im Testleiter-Client. 