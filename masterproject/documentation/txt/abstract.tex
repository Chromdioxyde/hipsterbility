% ----------------------------------------------------------------------------------------------------------
% Abstract
% ----------------------------------------------------------------------------------------------------------
\setcounter{page}{1}
\onehalfspacing
%\titlespacing{\section}{0pt}{12pt plus 4pt minus 2pt}{2pt plus 2pt minus 2pt}
\rhead{KURZFASSUNG / ABSTRACT}


\section*{Kurzfassung}
Dieses Projekt befasst sich mit der Konzeption und Realisierung eines Usablility Testing Frameworks für die Android Plattform.
Die Basis bietet ein bestehender Prototyp, der systematisch erweitert wird.
Kernstück ist eine Android Bibliothek, welche es ermöglicht den Bildschirminhalt, das Kamerabild der Frontkamera und weitere Daten zu erfassen.
Diese Daten werden erfasst, währen der Benutzer eine vorgegebene Liste von Testaufgaben nacheinander bearbeitet.
Am Ende der Sitzung werden die Daten zu der zweiten Komponente, dem Server hochgeladen, wo sie der Testleiter mit einem Client abrufen kann, dritte Komponente.

Das Ergebnis soll eine Darstellung des Bildschirminhalts und das Kamerabild der Frontkamera umfassen.
Ziel des Projekts ist ein Werkzeug um Remote Usability Testing Studien durchführen zu können.
Diese Ziel wurde jedoch nicht in vollem Umfang erreicht.

%\vspace{-1,2em}
%\titlespacing{\section}{0pt}{12pt plus 4pt minus 2pt}{-6pt plus 2pt minus 2pt}
%\section*{Abstract}

