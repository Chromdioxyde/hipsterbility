\subsection{Server}
Die Arbeit an der Serverkomponente war die umfangreichste am Projekt.
Probleme mit der \emph{Glassfish} Konfiguration und dessen teilweise unzureichende Dokumentation haben sehr viel Zeit in Anspruch genommen.
Das Ergebnis ist ein Server mit Datenbankanbindung und \ac{REST}-API, die weitgehend implementiert wurde.
Einzig Funktionen zum Dateiupload konnten aus Zeitgründen nicht mehr implementiert werden, wodurch die Android Bibliothek die gesammelten Daten nicht hochladen kann und die Daten entsprechend nicht für die Auswertung bereitstehen.

\subsection{Testleiter-Client und Android Bibliothek}
Umgesetzt wurden die Anmeldung und die Benutzerverwaltung.
Die Kommunikation mit dem Server wurde in Ansätzen implementiert, verläuft aktuell jedoch synchron im UI-Thread, was zum Blockieren der \ac{GUI} führt, wenn große Datenmengen abgerufen werden.
Um dies zu verhindern sollte der Datentransfer in JavaFX Tasks nebenläufig durchgeführt werden, dies konnte jedoch in der vorgegebenen maximalen Bearbeitungsdauer nicht mehr umgesetzt werde.

Die Android Bibliothek wurde ein wenig verbessert, bei der Erkennung von Eingaben und Gesten und die Persistenzschicht wurde eingeführt, ist jedoch noch nicht voll implementiert.
Es wurde viel Zeit investiert, um eine bessere Möglichkeit zu finden den Bildschirminhalt zu erfassen, ohne Root-Zugriff das System zu haben.
Dies verlief jedoch weitgehend ergebnislos.
Zwar wäre es möglich eine Systemapplikation zu erstellen, dies würde allerdings die Integration in andere Anwendungen erschweren, da Systemanwendungen eine besondere Signatur erfordern. 