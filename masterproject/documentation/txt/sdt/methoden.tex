\subsection{Mobile Usability Labortests}
Usability-Tests können grob in zwei räumliche Kategorien eingeteilt werden, Labortests und Feldtests, welche jeweils moderiert oder unmoderiert stattfinden können.

Ein einfacher Testaufbau für Labortests lässt sich z.B. mit Hilfe eines Computers mit angeschlossener, externer Kamera realisiert werden. 
Vorausgesetzt werden auch entsprechende Räumlichkeiten und Benutzer, welche die Tests durchführen.
Die, an den Computer angeschlossene Kamera, dient zum Aufzeichnen und Dokumentieren des Bildschirms eines mobilen Gerätes auf dem getestet wird.
Über die Kamera werden auch die Benutzereingaben dokumentiert. \cite[Vgl.][]{Budiu.2014}
Dies ist ein Beispiel für gängige Labortests.
Testmethoden für Labortests werden an dieser Stelle nicht weiter behandelt.
Im nächsten Abschnitt werden zwei Anbieter für Remote Usability Tests vorgestellt.
