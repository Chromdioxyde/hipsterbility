\subsection{Mobile Usability Testing Anbieter\label{subsec:usability_testing_anbieter}}
Neben Methoden und Werkzeugen zum Messen und Testen von Software-Usability gibt es auch Dienstleister, deren Geschäftsmodell auf Usability-Tests und Untersuchungen basiert.
Hier werden einige Anbieter solcher Dienstleister kurz vorgestellt und das Angebot kurz beschrieben.
Die Auswahl und Beschreibung beschränkt sich auf Anbieter, die das Testen von mobile Applikationen anbieten.

\subsubsection{UserTesting\label{subsubsec:usertesting}}
Der Dienstleister \emph{UserTesting} (\url{http://www.usertesting.com/}) bietet, unter anderem, Usability-Studien an, welche von Testteilnehmern durchlaufen werden.
Das Angebot beinhaltet auch das Testen von mobilen Webseiten und Applikationen. \cite[Vgl.][]{UserTesting.2014c}

Der Kunden/Auftraggeber stellt Testaufgaben zusammen, welche die Testpersonen erfüllen sollen.
Es gibt die Möglichkeit Geräteklasse, Anzahl der Testteilnehmer und eine Zielgruppe festzulegen.
Das Ergebnis der in Auftrag gegebenen Studie kann, laut Anbieter, schon nach ca. einer Stunde vorliegen und umfasst Videos von Testteilnehmern währen der Tests, schriftliche Antworten und die Möglichkeit der anschließenden Befragung der Teilnehmer. \cite[Vgl.][]{UserTesting.2014b}

Das Angebot zum Testen von mobilen Applikationen umfasst die Plattformen Android und iOS.
Pro Plattform werden als Testgeräte Tablets und Smartphones angeboten, welche bei der Spezifikation einer Teststudie angegeben werden. 
Auch hier werden Zielgruppen ausgewählt und Testaufgaben spezifiziert.
Das Testergebnis umfasst auch eine Videoaufzeichnung vom Test, auf dem der Gerätebildschirm sichtbar ist und der Benutzer Aussagen nach der \emph{Thinking Aloud} Methode (siehe Abschnitt \ref{subsec:thinking_aloud}) tätigt.
Die pauschalen Kosten für kleine Studien belaufen sich auf \$49 pro Testbenutzer.
\cite[Vgl.][]{UserTesting.2014}

\subsubsection{UserZoom \label{subsubsec:userzoom}}
Das \emph{UserZoom} (\url{http://www.userzoom.de}) Angebot vereint Werkzeuge und Dienstleistungen.
Letztere sind vorwiegend unterstützender Natur und umfassen z.B. Beratung, Rekrutierung von Teilnehmern und den Kunden-Support \cite[vgl.][]{UserZoomGmbH.2013b}.

Das Angebot bzgl. des Mobile Testing umfasst eine mobile Applikation für die iOS und Android Plattformen, welche sich zum Testen von Webseiten und webbasierten App-Prototypen eignet.
Das Verfahren wird als \enquote{[...] Remote Unmoderated Mobile Usability Testing [...]} \cite{UserZoomGmbH.2013c} bezeichnet und erlaubt ein standortunabhängiges Testen.
Angemeldete Testpersonen werden in Zielgruppen eingeteilt und per E-Mail zu Studien und Befragungen eingeladen.
Das eigentliche Testen wird über eine native mobile Applikation durchgeführt, in welcher Fragen beantwortet und Aufgaben erfüllt werden sollen.
Mit der Applikation lassen sich keine nativen Applikationen testen.
\cite[Vgl.][]{UserZoomGmbH.2013c}

Im Gegensatz zu \emph{UserTesting} gibt \emph{UserZoom} keine Pauschalpreise an.
Ein Rechenbeispiel auf der Webseite des Unternehmens vergleicht Remote Usability Testing mit Labortests.
Dabei werden Kosten von bis zu 120~\texteuro\xspace pro Testbenutzer in Labortests und ca. 10~\texteuro\xspace für Remote-Testbenutzer berechnet.
Die Kostengegenüberstellung am Ende des Artikels, welche eine hypothetische Ersparnis von 40~\% berechnet lässt mit 12\,440~\texteuro\xspace zu 7600~\texteuro\xspace darauf schließen, dass sich das Angebot eher an Unternehmen richtet, als an einzelne Entwickler oder kleine Teams.
\cite[Vgl.][]{UserZoomGmbH.2013d}