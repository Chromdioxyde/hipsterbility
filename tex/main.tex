%
%
\documentclass[11pt,headsepline,footsepline,plainfootsepline]{scrartcl}

% own geometry
% siehe Skript "Leitfaden Wissenschaftliches Arbeiten Oktober 2008" WiSo S.52
% oben und unten zusätzliche Korrekturen für Header und Footer + Binderand links
\usepackage[a4paper, top=3.5cm, left=3cm, right=2cm, bottom=3.5cm]{geometry}


\usepackage[ngerman]{babel} 
\usepackage[utf8]{inputenc} 
\usepackage[T1]{fontenc}
\usepackage{graphicx}
\usepackage{color}
\usepackage{xcolor}
\usepackage{wrapfig} %von Text umflossene Bilder
%\usepackage{jurabib}
\usepackage{csquotes}
%Links ohne Farbmarkierung und Rahmen, vor allem für Ausdruck.
\usepackage[colorlinks=false, pdfborder={0 0 0}]{hyperref}
\usepackage[printonlyused, footnote]{acronym}
\usepackage{dirtree}
\usepackage[tracking=true]{microtype}
%fonts
\usepackage{helvet}
\renewcommand{\familydefault}{\sfdefault}
\fontfamily{phv}\selectfont

%\include{lib/jurabib}
%\bibliographystyle{jurabib}
\usepackage[
	backend=biber,
	style=alphabetic,
	citestyle=alphabetic,
	sorting=nty,
	doi=true,url=true
]{biblatex}
\bibliography{bib/bib}

% Volle Namen der Autoren mit \citeauthor*{}
\DeclareCiteCommand*{\citeauthor}
{\defcounter{maxnames}{99}%
\defcounter{minnames}{99}%
\defcounter{uniquename}{2}%
\boolfalse{citetracker}%
\boolfalse{pagetracker}%
\usebibmacro{prenote}}
{\ifciteindex{\indexnames{labelname}}{}%
\printnames{labelname}}
{\multicitedelim}
{\usebibmacro{postnote}}

% setup of source code listings
\usepackage{listings}
%\usepackage{courier}
\usepackage{caption}
\lstset{
	basicstyle=\footnotesize\ttfamily,	% default font
	numbers=left,						% line numbers placement
	numberstyle=\tiny,					% line numbers style
	%stepnumber=2,						% line number padding
	numbersep=5pt,						% padding between line numbers and code
	tabsize=2,							% 
	extendedchars=true,         
	breaklines=true,					% line breaks 
	keywordstyle=\color{red},
	frame=b,
	stringstyle=\color{gray}\ttfamily,	% color of strings in code
	showspaces=false,					% visualize spaces
    showtabs=false,						% visualize tabs
    xleftmargin=17pt,
	framexleftmargin=17pt,
	framexrightmargin=5pt,
	framexbottommargin=4pt,
	showstringspaces=false				% visualize spaces in strings        
 }
 
 \lstloadlanguages{% Check docs for further languages ...
         C,
         C++,
         bash,
         HTML,
         Java,
         Clean % use this for CSS
 }

\setlength{\parindent}{0pt}
\setlength{\parskip}\medskipamount

\DeclareCaptionFont{white}{\color{white}}
\DeclareCaptionFormat{listing}{\colorbox{gray}{\parbox{\textwidth}{#1#2#3}}}
\captionsetup[lstlisting]{format=listing,labelfont=white,textfont=white}

% layout the caption ontop of code
\captionsetup[lstlisting]{format=listing,labelfont=white,textfont=white, singlelinecheck=false, margin=0pt, font={bf,footnotesize}}

% Headings
%\usepackage{fancyhdr}
%\renewcommand{\footrulewidth}{0.4pt}
%\cfoot{}
%\rfoot{\thepage}

\usepackage[olines]{scrpage2}
\clearscrheadfoot
\automark[section]{chapter}
\automark[subsection]{section}
\ihead[scrplain-innen] {\leftmark} %Linke Kofpzeile. Einseitig, deshalb ist innen links.
\ohead[scrplain-außen] {\rightmark} %Rechte Kopfzeile
\ofoot[scrplain-außen] {\pagemark}

% Document begins now
\begin{document}

\author{%
	Oliver Erxleben \small(\href{mailto:oliver.erxleben@hs-osnabrueck.de}{oliver.erxleben@hs-osnabrueck.de})\\%
	Albert Hoffmann \small(\href{mailto:albert.hoffmann@hs-osnabrueck.de}{albert.hoffmann@hs-osnabrueck.de})\\%
	\\%
	Hochschule Osnabr"uck \\%
	Ingenieurswissenschaften und Informatik \\%
	Informatik - Mobile und Verteilte Anwendungen\\
	Mensch-Maschine-Kommunikation - Wintersemester 2013/14 }

\title{\includegraphics[scale=0.75,keepaspectratio]{img/hs_os.png}\linebreak \linebreak Entwicklung eines Usability-Test-Frameworks für Android}


\maketitle
\thispagestyle{empty}
\pagestyle{empty}

\newpage
%\thispagestyle{empty}
\tableofcontents


\begingroup
%\thispagestyle{empty}
\newpage
\section*{Abkürzungsverzeichnis}

%
\begin{acronym}[JSON] %längste Abkürzung damit Abstände passen 
\acro{HTTP}{Hypertext Transfer Protocol}
\acro{SSL}{Secure Sockets Layer}
\acro{TLS}{Transport Layer Security}
\acro{UI}{User Interface}
\acro{JSON}{JavaScript Object Notation}
\acro{UX}{User Experience}
\acro{SDK}{Software Development Kit}
\acro{IDE}{Integrated Development Environment}
\acro{API}{Application Programming Interface}
\acro{NPM}{Node Package Manager}
\acro{VM}{Virtuelle Maschine}
\acro{SSH}{Secure Shell}
\acro{JAR}{Java Archive File}
\acro{CSS}{Cascading Style Sheets}
\acro{SQL}{Structured Query Language}
\end{acronym}
\nopagebreak
\listoffigures
%\thispagestyle{empty}
\lstlistoflistings
%\thispagestyle{empty}
\newpage
\endgroup

\thispagestyle{empty}

\begin{abstract}

\noindent\textbf{Zusammenfassung:}\\ \\
Die vorliegende Ausarbeitung wurde in \LaTeX verfasst und ist eine gemeinsame Arbeit von Albert Hoffmann und Oliver Erxleben an der Hochschule Osnabrück / University of Applied Sciences im Masterstudiengang Informatik - Mobile und Verteilte Anwendungen des Fachbereichs Ingenieurswissenschaften und Informatik für das Modul Mensch-Maschine-Kommunikation im Wintersemester 2013/14. 

Die Arbeit dokumentiert die Entwicklung einer Programmierbibliothek zum Test der Usability von Mobilanwendungen für das Android-Betriebssystem.

Das Ziel ist es, eine Bibliothek für das Android Betriebssystem zu schreiben, die Entwickler einfach in ihre Anwendungen integrieren können und mit der sich Usability Untersuchungen durchführen lassen.

Zur Verwaltung wird für Entwickler/Testleiter eine webbasierte Oberfläche bereitgestellt, über die Test und Aufgaben verwaltet werden können. 
Das Ergebnis ist ein Video, welches das Kamerabild der Gerätefrontkamera und den Bildschirminhalt des Testgerätes enthält. Diese Video kann über die Verwaltungssoftware, nach Abschluss der Tests, Angezeigt werden.

Für die Kommunikation wird außerdem eine \ac{API} entwickelt, welche von der Verwaltungsoberfläche und der Android Bibliothek genutzt wird.
Alle Komponenten werden als Prototypen entwickelt und sind eher als Proof-of-Concept anzusehen, entsprechend eignen sich die frühen Versionen weniger für den Produktiveinsatz.

\end{abstract}

\newpage
% set new page style

%\pagestyle{fancy}
\pagestyle{scrheadings}
\setcounter{page}{1} 

% Einleitung 
\section{Motivation}

Mobile Anwendungen, sowie mobile Versionen einer Web-Site oder Web-Anwendung sind seit dem Aufkommen von Smartphones und Tablets im täglichen Leben unserer Gesellschaft verankert. Diese Anwendungen, allgemeinhin als \textit{Apps} bezeichnet, sollen uns im Alltag helfen, vernetzen, navigieren oder gar zerstreuen. Die Möglichkeiten der Smartphone-Hardware erweitern sich mit jeder Revision. Neue Programmierschnittstellen und präzisere Sensoren ermöglichen eine stetige Verbesserung und neue Nutzungsmöglichkeiten für mobile Endgeräte. 

Neben den Funktionen die durch Apps bereitgestellt werden, müssen diese aber auch leicht zu benutzen sein. Jede Funktion kann noch so hilfreich sein, wenn sie nur schlecht bedienbar ist. 

Das Thema \textit{Usability} spielt für die Software-Entwicklung eine immer größere Rolle. Eine Anwendung kann sich nur gut verkaufen, wenn sie auch bedienbar oder benutzerfreundlich ist, wenn sie sich "richtig anfühlt". Usability ist aber auch kein neues Thema im Software-Engineering. Seit Jahren beschäftigen sich Menschen mit der Verbesserung der Usability von Anwendungen.

Es werden Labor- oder Feldtests durchgeführt, Dienstleistungen werden angeboten und es werden Testprotokolle aufgezeichnet. Beispiele für solche Dienstleister sind: 

\begin{itemize}
    \item{http://www.userfeedbackhq.com/}
    \item{http://rapidusertests.com/}
    \item{http://www.usertesting.com/mobile}
\end{itemize}

Die Dienstleister erstellen mit Testsubjekten Anwendungstests und geben ein Feedback in Form von Video-, Audio- und Textmaterial. Dazu werden unterschiedliche Techniken eingesetzt. Zum Beispiel die Aufnahme des Smartphones über eine Helmkamera. 

Die gelisteten Dienste prüfen eine fertige App, einen Prototyp oder einen bestimmten Entwicklungsstand einer Anwendung. Usability-Tests sind aufwändig und teuer. Je nach Zielgruppe der Anwendung können hohe Kosten für einen Test entstehen. Es müssen geeignete Tester gefunden und es muss Hardware beschaffen werden. Zudem kann das Testen sehr zeitintensiv werden. %TODO: negative Punkte an den Diensten

Es stellt sich die Frage, wie Usability bereits während der Entwicklung getestet werden kann und wie eine Bibliothek, bzw. ein Framework, Funktionen zum Usability-Test bereitstellen kann. Welche Funktionen muss eine Bibliothek aufweisen um Usability zu testen? Dieser Frage widmet sich die vorliegende Ausarbeitungen. %TODO: überarbeiten/Erweitern!

%example citings
%\cite{usabilityblog_eResult}

%\cite{usabilityblog_wasBeachten}

%\cite{usabilityEngineeringKompakt} %
%\newpage
% \newpage statt \pagebreak für neue Abschnitte, siehe:
% http://tex.stackexchange.com/questions/736/pagebreak-vs-newpage
\section{Testen von Usability}
\label{usability_testing}

Dieser Abschnitt stellt mögliche Vorgehen beim Testen von Usability vor, grenzt sie voneinander ab und bewertet sie für die Implementierung der Bibliothek. 

% TODO: Testarten, Methoden, Abgrenzung, .... Theorie und Transferleistung

\cite{usabilityblog_wasBeachten}

\cite{usabilityblog_eResult}
\section{Anwendungsarchitektur}
\label{app_architecture}
Dieser Abschnitt beschreibt das implementierte System in der Gesamtheit. Es werden alle Komponenten client- und serverseitig vorgestellt. Die darauffolgenden Abschnitte widmen sich bestimmten Kernkomponenten detailliert. 

Das zu implementierende Werkzeug kann in zwei Bereiche gegeliedert werden, dem Android-Client und dem HTTP-Server. Der Server

\subsection{Client-Server-Kommunikation}


%TODO: write description, system requirements, Architekturbild ...


\section{Server}
\label{server}
Wie in Abschnitt \ref{app_architecture} beschrieben, müssen die aufgezeichneten Daten der Clientbibliothek weiter verarbeitet werden. % nochmal etwas genauer

Im Folgenden werden die serverseitigen Komponenten, virtuelle Maschine, Datenbank und HTTP-Server, vorgestellt. 

%TODO: footnote ++ update


\subsection{Virtuelle Maschine}

\subsubsection{Konfiguration \label{sec:vm-config}}

Als Betriebssystem für den Server dient ein Debian in der Version 7.0. Der Maschine wurde 1 Prozessorkern und 512 MB RAM zugewiesen. Änderungen können im Vagrantfile, siehe Abschnitt \ref{vagrant}, oder der Virtual Box-Konfiguration vorgenommen werden. 

Folgende Software wurde für den Entwicklungsprozess verwendet. 
\begin{description}
	\item[OpenSSH-Server\footnotemark]\footnotetext{http://www.openssh.org/} 
Das Programmpaket für die \emph{Secure Shell} beinhaltet alle Softwaretools um Dateien zwischen Host- und Gast- System auszutauschen (scp, sftp) oder sich darauf zu verbinden (ssh). Das Softwarepaket ist notwendig um beispielsweise Vagrant (siehe Abschnitt \ref{sec:Vagrant}) betreiben zu können. 

	\item[MySQL Server\footnotemark]\footnotetext{http://www.mysql.de/products/community/}
Zur persistenten Speicherung der Nutzdaten wird MySQL verwendet. MySQL bietet sich an, da es Verbreitung und Sicherheit bietet, sowie von Oracle als kommerzielles, wie auch Open Source-Derivat entwickelt wird. Das Projekt nutzt die Community-Server-Edition.
	
	\item[Node.js\footnotemark und NPM\footnotemark]\footnotetext{http://nodejs.org/}\footnotetext{https://www.npmjs.org/}
Node.js, kurz Node, ist eine Platform basierend auf der JavaScript Laufzeitumgebung des Browsers Chrome. Die Intension von Node ist es schnelle, skalierbare Netzwerkanwendung entwickeln zu können. Dabei wird ein ereignisgetriebenes, nicht blockierendes Eingabe-/Ausgabesystem verwendet. Der Node Package Manager, kurz NPM, wird verwendet um weitere Abhängigkeiten der Anwendung zu verwalten.
	
\end{description}

\subsubsection{Vagrant \label{sec:Vagrant}}
\label{vagrant}
Um die Bereitstellung der Serverumgebung zu vereinfachen wurde die virtuelle Maschine mittels Vagrant\footnote{Vagrant: http://www.vagrantup.com/} konfiguriert. Vagrant ist ein Kommandozeilenwerkzeug um schnell reproduzierbare Entwicklungsumgebungen zu schaffen und diese später zu verteilen oder zu exportieren. Dabei wird Virtual Box\footnote{https://www.virtualbox.org/} als Virtualisierungssoftware eingesetzt. Aber auch VMware wird unterstützt. 

Die eigene VM wird mittels einer VM-Schablone (die eigentliche VM) und einer Konfigurationsdatei geladen, die alle Eigenschaften der VM bereithält, wie zum Beispiel Portweiterleitungen. Die Schablone kann demnach bereits einige Standardsoftwarepakete beinhalten. 

Auf dem Host-System können die gewohnten Entwicklungswerkzeuge eingesetzt werden, da der Ordner, indem Vagrant konfiguriert wurd, mit dem Ordner /Vagrant auf dem Gast-System synchronisiert wird. 

Auch die Entwicklung von Software mittels einer Vagrant VM ergibt einen gut zu bedienenden Arbeitsfluss. So ist es beispielsweise möglich mittels \textbf{vagrant up} und \textbf{vagrant ssh} die Maschine zu starten und darauf zu verbinden, ohne dabei extra overhead, wie zum Beispiel Fenster, zu erzeugen. Die SSH-Credentials folgen dem \textit{Convention over Configuration}-Paradigma. Username, sowie Passwort lauten standardmäßig vagrant. 

\subsection{Datenbank}

\subsubsection{Datenbank-Modell}
Das Datenbankmodell wurde mit Hilfe des MySQL Workbench erstellt. Die Abbildung \ref{figure-db-model} zeigt das Ergebnis.
\begin{figure}[h!]
	\centering
	\includegraphics[width=\linewidth,keepaspectratio]{img/db_model.png}
	\caption{Datenbank Modell}
	\label{figure-db-model}
\end{figure}


\subsection{HTTP-Server und Middleware}
Die serverseitige Logik wurde mit dem Express-Framework\footnote{http://expressjs.com/} umgesetzt. Folglich wurde serverseitig Javascript eingsetzt. Javascript eigenet sich sehr gut für Programmierschnittstellen und HTTP-Anfragen. Nicht nur, da Javascript aus dem Web nicht mehr wegzudenken ist, sondern auch da es komplett asynchron programmiert werden kann. 

Das Express-Framework stellt die grundlegenden Funktionen eines Web-, bzw. HTTP-Servers bereit. Ein minimaler HTTP-Server ist im Listing \ref{minimal_node_http_server} zu sehen. Eine Express-Anwendung stellt Pfade, sog. \emph{routes}, zur Verfügung auf die der HTTP-Server reagiert. Im Listing wird Zeile 23 wird eine solcher Pfad angelegt. 

\begin{lstlisting}[label=minimal_node_http_server,language=Java, caption=Minimaler Node-HTTP-Server]
/**
 * Module dependencies.
 */
var express = require('express');
var routes = require('./routes');
var http = require('http');
var path = require('path');
var app = express();

// environments config
app.set('port', process.env.PORT || 3000);
app.set('views', path.join(__dirname, 'views'));
app.set('view engine', 'ejs');
app.use(express.favicon());
app.use(express.logger('dev'));
app.use(express.json());
app.use(express.urlencoded());
app.use(express.methodOverride());
app.use(app.router);
app.use(express.static(path.join(__dirname, 'public')));

// simple test route
app.get('/ping', routes.pong);


// server take off
http.createServer(app).listen(app.get('port'), function(){
  console.log('Express server listening on port ' + app.get('port'));
});
\end{lstlisting}

Im Projektordner befindet sich der Unterordner \textbf{serverside}. In diesem finden sich alle Serverdateien. Die Baumstruktur aus Abbildung \ref{fig:server-tree} zeigt die Ordnerhierachie des Projekts.

Im Ordner \textit{node\_modules} befinden sich alle weiteren Module die für die Serveranwendung benötigt und durch NPM verwaltet werden (Details zu den Modulen finden sich im Abschnitt \ref{sec:node-mudules}). 

Im Unterordner \textit{controllers} werden die Web- und API-Routen-Handler gehalten. Diese werden in \textbf{app.js} als Callback-Funktionen aufgerufen (der Abschnitt \ref{sec: api} beschreibt die Routen detailliert). Im Ordner \textit{public} werden alle Dateien für die Browseranwendung (clientseitiges Javascript, CSS, Bilder und Schriftarten) angelegt und gehalten. Letztlich werden die HTML-Views im Unterordner \textit{views} gehalten. 

Die Ordnerstruktur entspricht weitestgehend der Standardkonfiguration einer Express-Web-Anwendung.

\begin{figure}[h!]
	\centering
			\begin{minipage}[c]{\textwidth} %Ordnerstruktur in Minipage damit die zusammengehalten wird
				\dirtree{%
				 .1 /serverside.
				 .2 classes.
				 .2 controllers.
				 	.3 api.
				 .2 node\_modules.
				 .2 public.
				 	.3 css.
				 	.3 fonts.
				 	.3 img.
				 	.3 js.
				 .2 uploads.
				 .2 views.
				}
			\end{minipage}
	\caption{Ordnerstruktur des Servers}
	\label{fig:server-tree}
\end{figure}

Der Node-Server kann mittels des Befehls \textbf{node app.js}, bzw. \textbf{npm start}, ausgeführt werden und lauscht, wenn nicht anders angegeben, auf den Port 3000. 

\subsubsection{Node Module \label{sec:node-mudules}}

Wie im Abschnitt \ref{sec:vm-config} beschrieben werden alle Abhängigkeit für den Node-Server mittels NPM verwaltet. NPM funktioniert wie andere Paketmanager. Auf Im Normalfall wird NPM mit Node ausgeliefert und installiert. 

Im Ordner \textit{serverside} befindet sich die Datei \emph{package.json}. Diese Datei enthält alle benötigten Abhängigkeiten im JSON\footnote{http://www.json.org/}-Format. Das Listing \ref{node-packages} zeigt das Paket-JSON-Objekt. Neben den Projektinformationen beinhaltet es ein weiteres Objekt, \emph{dependencies}. 

In diesem sind die jeweiligen Abhängigkeiten zu den verwendeten Modulen, zum Beispiel das Express-Framework, enthalten. Jeder Eintrag besteht aus dem Namen des Moduls und der gewünschten Version. Das \emph{*} bedeutet das immer die aktuellste Version genutzt werden soll. Mittels des Befehls \textbf{npm install} im Ordner \textit{serverside} können die Abhängigkeiten installiert werden. 

\begin{lstlisting}[label=node-packages,language=Java, caption=Abhängigkeiten der Node-Anwendung]
{
    "name": "hipsterbility-server",
    "version": "0.0.1",
    "private": true,
    "scripts": {
    "start": "node app.js"
    },
    "dependencies": {
    "express": "*",
	"jade": "*",
	"mysql" : "*",
    "avconv":"*",
    "passport":"*",
    "passport-local":"*",
    "string":"*",
    "mkdirp":"*"

  }
}
\end{lstlisting}

\subsubsection{Webbasiertes Verwaltungstool}

Aus dem 

\subsubsection{API \label{sec: api}}

\subsection{Videoproduktion}

Zur Erstellung der Zwischenergebnisse und des Endergebnisses wird die Open Source-Bibliothek \emph{libav}\footnote{https://libav.org/} verwendet. Genauer wird die darin enthaltene Software \emph{avconv} genutzt um die Bilder und Videos zusammenzufügen. 

\section{Android-Client}

In diesem Abschnitt wird die Entwicklung der Android-Bibliothek beschrieben. Zur Entwicklung wurde die IDE \textit{IntelliJ Idea}\footnote{IntelliJ Idea: http://www.jetbrains.com/idea/} verwendet. Screenshots und Anleitungen beziehen sich diesbezüglich auf diese Software. Konfigurationen und Einstellungen können sich zu anderen IDEs unterscheiden. 

Die Quellen zu den Projekten befinden sich im Github-Repository im Unterordner \textit{client}.

\subsection{Bibliotheksmodul}

Zur Entwicklung der Bibliothek wurden zwei Projekte verwendet - Einmal das Projekt für die Bibliothek selbst und weiterhin ein Android-Projekt zum Testen der Funktionen während der Entwicklung. 

\subsubsection{Monolithischer Wrapper}

\subsection{Einbinden der Bibliothek in eigene Anwendungen}

\subsubsection{Dynamisches Einbinden - Pre-Dexing}

Beim Pre-Dexing können mehrere Android-Projekte zu einem Projekt hinzugefügt werden. 
% TODO: updating

Da die Bibliothek neueste Funktionen des Android-Betriebssystems verwendet, muss auf dem  Gerät zur Kompilierung einer Anwendung aktuelle Build-Tools der Android-SDK vorliegen. Im Falle der Bibliothek ist dies API-Level 19. Offenbar hat sich hier ein Bug eingeschlichen und es ist unter der Version 19.0 kein Pre-Dexing möglich. Dies wurde allerdings bereits behoben. Es sei gesagt, dass mindestens die Version 19.0.1 verwendet werden muss um Pre-Dexing durchführen zu können. 
% TODO: update, wichtig? Screenshot? 

\subsubsection{Statisches Einbinden als JAR-Archiv}

%\include{content/messdaten}
\section{Ergebnis}

Als Testergebnise wurde der Arbeit ein beispielhaftes Video beigefügt, welches mit dem implementierten Werkzeug erstellt wurde. Dies kann unter % TODO: einfügen
\include{content/offenePunkte}

% More content goes here

%\renewcommand*{\biburlprefix}{(URL: }
%\renewcommand*{\biburlsuffix}{)}

\newpage
\clearscrheadfoot % Header und Footer entfernen
\ofoot[scrplain-außen] {\pagemark} % Seitenzahl unten rechts

\addcontentsline{toc}{section}{Literaturverzeichnis} % Eintrag ins Inhaltsverzeichnis
\printbibliography

\newpage
%\pagestyle{fancyplain}

\begin{appendix}
\section{Anhang}
\subsection{Android Testapplikation Bildschirme und Activities \label{anhang:test_app_screens}}
\begin{figure}[htb]
	\centering
	\includegraphics[width=0.75\linewidth]{img/client_test_app_screens}
	\caption{Bildschirme der Testapplikation mit farblich markierten Activities. \label{bild:client_test_app_screens}}
\end{figure}

\end{appendix}

\end{document}
	
