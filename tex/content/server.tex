\section{Server}
\label{server}

Im Folgenden werden die serverseitigen Komponenten, virtuelle Maschine, Datenbank und \ac{HTTP}-Server, vorgestellt. 

\subsection{Virtuelle Maschine}

\subsubsection{Konfiguration \label{sec:vm-config}}

Als Betriebssystem für den Server dient ein Debian in der Version 7.0. Der Maschine wurde 1 Prozessorkern und 512 MB RAM zugewiesen. Änderungen können im Vagrantfile, siehe Abschnitt \ref{vagrant}, oder der Virtual Box-Konfiguration vorgenommen werden. 

Folgende Software wurde für den Entwicklungsprozess verwendet. 
\begin{description}
	\item[OpenSSH-Server\footnotemark]\footnotetext{http://www.openssh.org/} 
	Das Programmpaket für die \emph{Secure Shell} beinhaltet alle Softwaretools um Dateien zwischen Host- und Gast- System auszutauschen (scp, sftp) oder sich darauf zu verbinden (ssh). 
	Das Softwarepaket ist notwendig um beispielsweise Vagrant (siehe Abschnitt \ref{sec:Vagrant}) betreiben zu können. 

	\item[MySQL Server\footnotemark]\footnotetext{http://www.mysql.de/products/community/}
	Zur persistenten Speicherung der Nutzdaten wird MySQL verwendet. MySQL bietet sich an, da es ausreichende Verbreitung und Sicherheit bietet, sowie von Oracle als kommerzielles, wie auch Open Source-Derivat entwickelt wird. Dieses Projekt nutzt die Community-Server-Edition.
	
	\item[Node.js\footnotemark und NPM\footnotemark]\footnotetext{http://nodejs.org/}\footnotetext{https://www.npmjs.org/}
	Node.js, kurz Node, ist eine Plattform basierend auf der JavaScript Laufzeitumgebung des Browsers Chrome. 
	Die Intention von Node ist es schnelle, skalierbare Netzwerkanwendung entwickeln zu können.
	Dabei wird ein ereignisgetriebenes, nicht blockierendes Eingabe-/Ausgabesystem verwendet.
	Die von Node verwendete Paketverwaltung, kurz \emph{npm}, wird verwendet um weitere Abhängigkeiten der Anwendung zu verwalten.
	
\end{description}

\subsubsection{Vagrant \label{sec:Vagrant}}
\label{vagrant}
Um die Bereitstellung der Serverumgebung zu vereinfachen wurde die virtuelle Maschine mittels Vagrant\footnote{Vagrant: http://www.vagrantup.com/} konfiguriert. 
Vagrant ist ein Kommandozeilenwerkzeug um schnell reproduzierbare Entwicklungsumgebungen zu schaffen und diese später zu verteilen oder zu exportieren. 
Dabei wird Virtual Box\footnote{https://www.virtualbox.org/} als Virtualisierungssoftware eingesetzt. 
Alternativ wird auch VMware unterstützt. 

Die eigene \ac{VM} wird mittels einer VM-Schablone (die eigentliche VM) und einer Konfigurationsdatei geladen, die alle Eigenschaften der VM bereithält, wie zum Beispiel Portweiterleitungen. 
Die Schablone kann demnach bereits einige Standardsoftwarepakete beinhalten. 

Auf dem Host-System können die gewohnten Entwicklungswerkzeuge eingesetzt werden, da der Ordner, indem Vagrant konfiguriert wurd, mit dem Ordner /Vagrant auf dem Gast-System synchronisiert wird. 

Auch die Entwicklung von Software mittels einer Vagrant VM ergibt einen gut zu bedienenden Arbeitsfluss. 
So ist es beispielsweise möglich mittels \textbf{vagrant up} und \textbf{vagrant ssh} die Maschine zu starten und darauf zu verbinden, ohne dabei extra \enquote{overhead}, wie zum Beispiel Fenster, zu erzeugen. 
Die SSH-Credentials folgen dem \textit{Convention over Configuration}-Paradigma. 
Benutzername, sowie Passwort lauten standardmäßig vagrant. 

\subsection{Datenbank}

Das Datenbankmodell wurde, den Anforderungen entsprechend, mit Hilfe des MySQL Workbench erstellt.
Einige der Tabellen dienen zum Skizzieren möglicher Funktionalität und wurden noch nicht vollständig vom Datenmodell in die Anwendungen umgesetzt (\texttt{devices, apps, logs, audios}).
Bei späterer Erweiterung der Applikationen um o.g. Daten muss das Datenmodell nicht verändert werden.

Die Abbildung \ref{figure-db-model} zeigt das Ergebnis der Umsetzung des logischen Datenmodells in MySQL Datentypen.
Da in MySQL der kein Boolean-Datentyp vorhanden ist, wurde an dessen Stelle \texttt{TINYINT} verwendet, welcher entsprechend interpretiert wird.

Für das Erstellen von Tabellen und Beziehungen wurde ein \ac{SQL}-Skript erstellt (\texttt{init.sql}).
Diese ist auch im Installationsskript \texttt{install.sh} enthalten.
Zu Demonstrationszwecken wurde außerdem ein Skript mit Beispieldaten (hipsterbility\_test\_data.sql) hinzugefügt.
Beide Dateien sind im Ordner \texttt{serverside} zu finden.


\subsection{HTTP-Server und Middleware}
Die serverseitige Logik wurde mit dem Express-Framework\footnote{http://expressjs.com/} umgesetzt. 
Folglich wurde für die Programmierung Javascript eingesetzt. Die Sprache eignet sich sehr gut für Programmierschnittstellen und HTTP-Anfragen. 
Nicht nur, da Javascript aus dem Web nicht mehr wegzudenken ist, sondern auch da es komplett asynchron programmiert werden kann. 

Das Express-Framework stellt die grundlegenden Funktionen eines Web-, bzw. HTTP-Servers bereit.
Ein minimaler HTTP-Server ist im Listing \ref{minimal_node_http_server} zu sehen. 
Eine Express-Anwendung stellt Pfade, sog. \emph{routes}, zur Verfügung auf die der HTTP-Server reagiert. Im Listing wird Zeile 23 wird eine solcher Pfad angelegt. 

\begin{lstlisting}[label=minimal_node_http_server,language=Java, caption=Minimaler Node-HTTP-Server]
/**
 * Module dependencies.
 */
var express = require('express');
var routes = require('./routes');
var http = require('http');
var path = require('path');
var app = express();

// environments config
app.set('port', process.env.PORT || 3000);
app.set('views', path.join(__dirname, 'views'));
app.set('view engine', 'ejs');
app.use(express.favicon());
app.use(express.logger('dev'));
app.use(express.json());
app.use(express.urlencoded());
app.use(express.methodOverride());
app.use(app.router);
app.use(express.static(path.join(__dirname, 'public')));

// simple test route
app.get('/ping', routes.pong);


// server take off
http.createServer(app).listen(app.get('port'), function(){
  console.log('Express server listening on port ' + app.get('port'));
});
\end{lstlisting}

Im Projektordner befindet sich der Unterordner \textbf{serverside}. In diesem finden sich alle Serverdateien. 
Die Baumstruktur aus Abbildung \ref{fig:server-tree} zeigt die Ordnerhierachie des Projekts.

Im Ordner \textit{node\_modules} befinden sich alle weiteren Module die für die Serveranwendung benötigt und durch NPM verwaltet werden (Details zu den Modulen finden sich im Abschnitt \ref{sec:node-mudules}). 

Im Unterordner \textit{controllers} werden die Web- und API-Routen-Handler abgelegt. 
Diese werden in \textbf{app.js} als Callback-Funktionen aufgerufen (der Abschnitt \ref{sec: api} beschreibt die Routen detailliert). 
Im Ordner \textit{public} werden alle Dateien für die Browseranwendung (clientseitiges Javascript, CSS, Bilder und Schriftarten) angelegt und gehalten. 
Letztlich werden die HTML-Views im Unterordner \textit{views} abgelegt. 

Die Ordnerstruktur entspricht weitestgehend der Standardkonfiguration einer Express-Web-Anwendung.

\begin{figure}[h!]
	\centering
			\begin{minipage}[c]{\textwidth} %Ordnerstruktur in Minipage damit die zusammengehalten wird
				\dirtree{%
				 .1 /serverside.
				 .2 classes.
				 .2 controllers.
				 	.3 api.
				 .2 node\_modules.
				 .2 public.
				 	.3 css.
				 	.3 fonts.
				 	.3 img.
				 	.3 js.
				 .2 uploads.
				 .2 views.
				}
			\end{minipage}
	\caption{Ordnerstruktur des Servers}
	\label{fig:server-tree}
\end{figure}

Der Node-Server kann mittels des Befehls \textbf{node app.js}, bzw. \textbf{npm start}, ausgeführt werden und lauscht, wenn nicht anders angegeben, auf den Port 3000. 

\subsubsection{Node Module \label{sec:node-mudules}}

Wie im Abschnitt \ref{sec:vm-config} beschrieben werden alle Abhängigkeit für den Node-Server mittels \emph{npm} verwaltet. 
\emph{npm} funktioniert ähnlich wie andere Paketmanager. 
Im Normalfall wird \emph{npm} mit Node ausgeliefert und installiert. 

Im Ordner \textit{serverside} befindet sich die Datei \emph{package.json}. Diese Datei enthält alle benötigten Abhängigkeiten im JSON\footnote{http://www.json.org/}-Format. Das Listing \ref{node-packages} zeigt das Paket-JSON-Objekt. 
Neben den Projektinformationen beinhaltet es ein weiteres Objekt, \emph{dependencies}. 

In diesem sind die jeweiligen Abhängigkeiten zu den verwendeten Modulen, zum Beispiel das Express-Framework, enthalten. Jeder Eintrag besteht aus dem Namen des Moduls und der gewünschten Version. Das \textbf{*} bedeutet das die aktuellste Version genutzt werden soll. Mittels des Befehls \textbf{npm install} im Ordner \textit{serverside} können die Abhängigkeiten installiert werden. 

\begin{lstlisting}[label=node-packages,language=Java, caption=Abhängigkeiten der Node-Anwendung]
{
    "name": "hipsterbility-server",
    "version": "0.0.1",
    "private": true,
    "scripts": {
    "start": "node app.js"
    },
    "dependencies": {
    "express": "*",
	"jade": "*",
	"mysql" : "*",
    "avconv":"*",
    "passport":"*",
    "passport-local":"*",
    "string":"*",
    "mkdirp":"*"

  }
}
\end{lstlisting}

\subsubsection{Request-Response-Ablauf \label{sec:req-res-cycle}}

Um den Zyklus einer HTTP-Anfrage und -Antwort zu veranschaulichen wird im Folgenden ein beispielhafter Ablauf gezeigt. Als Beispiel dient die Auslieferung wird das \emph{Holen einer speziellen Session}.  

\begin{lstlisting}[label=node-getSessionRoute,language=Java, caption=Routenlistener für eine spezielle Session, firstnumber=169] 
app.get('/:user_id/sessions/:session_id/?', sessions.get);
\end{lstlisting}

Die Ressource kann durch die Implementierung der Route aus dem Listing \ref{node-getSessionRoute} angefragt werden. Der erste Parameter der Funktion \textbf{.get} bestimmt die Route. Anhand dieser Route wird die Zuordnung zum Benutzer (\emph{:user\_id}) gewährleistet und über den Parameter \emph{:session\_id} wird die spezielle Session zugewiesen. Der zweite Parameter ist der sog. Callback. Dieser Funktion wird von Express aufgerufen sobald die Route zutrifft um eine Antwort zu generieren. In diesem Beispiel wird die Funktion \textbf{.get} vom Sessions-Objekt aufgerufen.

\begin{lstlisting}[label=node-getSessionHandler,language=Java, caption=Routenbehandlung für eine spezielle Session, firstnumber=29]
exports.get = function(req, res) {
	
	var qstr = 'SELECT * FROM sessions WHERE users_idusers = ' + req.params.user_id + ' AND idsessions = ' + req.params.session_id;
	var query = new Query;
	
	query.execute(qstr, function(rows){
		res.send(rows);
	});
	
};
\end{lstlisting}

Das Listing \ref{node-getSessionHandler} zeigt die Implementierung der Behandlungsroutine für eine bestimmte Session. Der Funktion werden zwei Parameter übergeben. Das \emph{req}-Objekt (Request) beinhaltet alle Informationen der Anfrage und das \emph{res}-Objekt (Response) alle Informationen für die Antwort. Über das Objekt \emph{req.params} kann auf die Benutzer-ID und die Session-ID zugegriffen werden. In diesem Beispiel werden beide Parameter in der SQL-Abfrage genutzt um die Zeile aus der Sessions-Tabelle zu holen, welche alle Informationen zu dieser Session beinhaltet (Zeile 31). 

Nachdem die SQL-Anfrage abgesetzt wurde und ggf. Daten enthält, werden diese dem Response-Objekt übergeben und die Methode \textbf{.send} wird aufgerufen, welche den Request-Response-Zyklus beendet und die Daten im JSON-Format ausliefert. 

\subsection{Webbasiertes Verwaltungstool \label{sec:webtool}}

\label{sec:web_verwaltung}
Nach Abschnitt \ref{sec: zielsetzung} ist es Ziel eine zentrale Komponente zur Verfügung zu stellen mit der Tests verwaltet werden können. 
Zudem wurde ein rudimentäres Login-System implementiert. 
An dieser Stelle sei gesagt, dass die webbasierte Verwaltung nur skizziert ist. 
Eine vollständige, auf Sicherheit geprüfte und nach allen heutigen Webstandards aufbauende, Webanwendung steht nicht zur Verfügung. 
Dies ist im Rahmen dieser Semesterarbeit nicht entstanden und auch nicht das Anliegen dieser Arbeit. 
Vielmehr handelt es sich um einen Proof-Of-Concept. 

Aus Gründen der Lesbarkeit wurden die Screenshots des Verwaltungswerkzeugs in den Anhang \ref{anhang:screens-webtool} verlagert. 

Grundlegend wurden die Browserkomponenten der Web-Anwendung mittels jQuery\footnote{http://jquery.com/} und Bootstrap\footnote{http://getbootstrap.com/} erstellt.

Die Abbildung \ref{fig: index-page} zeigt die Startseite der Anwendung. 
Über den Link \emph{Login} kann zur Login-Maske gelangt werden, welche in Abbildung \ref{fig: login-page} zu sehen ist. 

Nach erfolgreichen Login erscheint das Dashbaord der Administration, welche zum Zeitpunkt der Abgabe keine Informationen bereitstellt. 
Im Administrationsbereich hat der User nunmehr die Möglichkeit eine Liste der erstellten Sessions zu sehen (siehe Abbildung \ref{fig: sessions-page}) oder eine neue Session anzulegen (Abbildung \ref{fig: session-create-page}). 

In der Listenansicht der Sessions erhält der User die Möglichkeit Details zur Session anzuzeigen, welches in der Abbildung \ref{fig: session-detail-page} zu sehen ist. 
Ein weitere Link \emph{zum Ergebnis} öffnet die Ergebnisseite der Session (Abbildung \ref{fig: session-result-page}).

\subsubsection{Web-Routen}

Dieser Abschnitt zeigt alle Routen des Express-Anwendung für die Webschnittstelle im Überblick. 
Die jeweiligen \ac{HTTP}-Methoden (GET, PUT, POST, DELETE, ...) werden vor der Route angegeben.

\begin{description}

	\item[GET /] liefert die Index-Seite.
	\item[GET /login] liefert die Login-Seite.

	\item[GET /web-auth] Umleitungsroute, ob ein User erfolgreich eingeloggt ist.
	\item[POST /web-auth] validiert das login-Formular und erstellt ggf. eine Session.
	\item[GET /:user\_id/admin] liefert die Index-Seite nach erfolgreichen Login.
	\item[GET /:user\_id/admin/sessions] liefert die Übersichtsseite der Sessions.
	\item[GET /:user\_id/admin/sessions/new] liefert die Fomularseite zum anlegen einer neuen Session.
	\item[POST /:user\_id/admin/sessions/new] validiert das gesendete Webforumlar und legt einen neuen Eintrag in der Datenbank an.
	\item[GET /:user\_id/admin/sessions/:session\_id] liefert die Detailseite einer Session.
	\item[GET /:user\_id/admin/sessions/:session\_id] liefert die Ergebnisseite einer Session.
	
	
\end{description}

\subsection{RESTful-API \label{sec: api}}

Neben der webbasierten Verwaltung benötigt der mobile Client eine einfache Anbindung an die Funktionen des Servers. 
Dafür eignen sich \ac{API}s sehr gut. 
Das Express-Framework bietet sich auch dafür sehr gut an. 
Nachfolgend werden die \ac{API}-Routen beschrieben und anschließend wird die Implementierung einer Route beispielhaft gezeigt. 
Sofern nicht anders angegeben sind die Antworten auf die \ac{API}-Resourcen im \ac{JSON}-Format. 
Routenteile beginnend mit einem \textbf{:} werden von Express als Parameter behandelt und stehen in Node als Variablen des Request-Objekts zur Verfügung. 

\subsubsection{Sessions API}

\begin{description}
	\item[GET /:user\_id/sessions] liefert die Liste aller Sessions eines Users.
	
	\item[GET /:user\_id/sessions/:session\_id] liefert die Details einer Session. 
	
	\item[PUT /:user\_id/sessions] ein Update auf eine Session. Sendet das Beenden einer Session und beginnt die Erstellung des Ergebnisvideos.
\end{description}


\subsubsection{Videos API}

\begin{description}
	\item[GET /:user\_id/:session\_id/videos] liefert alle Videos einer Session.
	
	\item[GET /:user\_id/:session\_id/videos/:id\_videos] liefert Details zu einem Video einer Session.
	
	\item[POST /:user\_id/:session\_id/videos] sendet ein neues Video zu einer Session.
\end{description}


\subsubsection{Captures API}

\begin{description}
	\item[GET /:user\_id/:session\_id/captures] liefert alle Screenshots einer Session.
	
	\item[GET /:user\_id/:session\_id/captures/:id] liefert Details zu einem Capture.
	
	\item[POST /:user\_id/:session\_id/captures] sendet ein neues Capture.

\end{description}

\subsubsection{ToDos API}

\begin{description}

	\item[GET /:user\_id/:session\_id/todos] liefert alle Todo-Listen einer Session.
	\item[GET /:user\_id/:session\_id/todos/:todo\_id] liefert Details zu einer bestimmten TODO-Liste.
	\item[POST /:user\_id/:session\_id/todos] Sendet eine neue, leere TODO-Liste zu einer Session.
	
	\item[PUT /:user\_id/:session\_id/todos/:todo\_id] aktualisiert eine TODO-Liste.
	
\end{description}

\subsubsection{Tasks API}

\begin{description}

	\item[GET /:user\_id/:session\_id/todos/:todo\_id/tasks] aktualisiert die Liste der Tasks einer TODO.
	\item[POST /:user\_id/:session\_id/todos/:todo\_id/tasks] sendet eine neue Task zu einer TODO.
	
	\item[GET /:user\_id/:session\_id/todos/:todo\_id/tasks/:task\_id] liefert Details zu einer Task.
	
	\item[PUT /:user\_id/:session\_id/todos/:todo\_id/tasks/:task\_id] aktualisiert eine Task.
	  
\end{description}

\subsection{Videoproduktion}

Zur Erstellung der Zwischenergebnisse und des Endergebnisses wird die Open Source-Bibliothek \emph{libav}\footnote{https://libav.org/} verwendet. 
Genauer wird das Kommandozeilenwerkzeug \textit{avconv} benötigt, um die Bilder und Videos zusammenzufügen. 
Weiterhin wird noch das Node Modul \emph{avconv} genutzt. 

Alle Daten werden am Client gesammelt und nach Beendigung einer Testsession an den Server via HTTP gesandt. 
Das Abschließen einer Session initiiert den Erstellungsvorgang der Videos. 
Die Erstellung des Ergebnisvideos geschieht in drei Teilen: 

\begin{itemize}
	\item{Jedes Bild wird anhand des Timestamps zu einem Video bestimmter länger konvertiert und anschließend konkateniert.}
	\item{Das aufgenommene Video der Frontkamera wird in seiner Breite gedoppelt.}
	\item{Das Zwischenergebnis der Screenshots und das aufgenommene Bild werden zu einem Endergebnis verbunden.}
\end{itemize}

Das Listing \ref{node-imgToVideo} zeigt die Implementierung der Konvertierung von Screenshot in die einzelnen Videos. 
Jeder Screenshot wird mit einem Zeitstempel (\texttt{timestamp}) als Dateinamen an den Server gesendet. 
Damit wird ein Screenshot einem Schlüssel-Wert-Paar in einem Array zugewiesen, welches durchlaufen wird. 
Es entsteht ein Video als Teilergebnis, welches nummeriert, beginnend mit Eins im mpeg-Format ausgegeben wird. 
Ein Screenshot wird für die Länge des Timestamps wiederholt im Teilergebnis angezeigt (Zeile 117).

Die Variable \textbf{stream} erzeugt ein \texttt{avconv}-Objekt und startet den Prozess (Zeile 132). 

\begin{lstlisting}[label=node-imgToVideo,language=Java, caption=Erstellung der Videos aus Screenshots, firstnumber=110]
for (var i = 0; i < paramDict.length; i++ ) {
	var output = './uploads/' + user + '/'+ session +'/results/captures/'+ (i+1)+'.mpg';
	var intervall = 0;
    
    if (i == paramDict.length-1) {
    		intervall = 4;
	} else {
		var tmp = Math.floor((parseInt(paramDict[i+1].stamp) - parseInt(paramDict[i].stamp)) / 1000);    
        intervall = tmp;
	}
	
	var input = './uploads/' + user + '/'+ session +'/captures/' + paramDict[i].filename;
	var paramProc = [
		'-f', 'image2',
		'-loop', '1',
		'-i', input,
		'-s', '720x1280',
		'-t', intervall,
		'-y',
		output
	];

	var stream = avconv(paramProc);
	stream.pipe(process.stdout);
	
	// ...
	
}
\end{lstlisting}

Sobald alle Teilergebnisse der Screenshots erstellt wurden, werden sie zu einem Gesamtvideo verbunden. 
Das Listing \ref{node-imgConcat} zeigt die Implementierung. 

Der Parameter \textit{concat:} erhält eine Liste mit Videopfaden die zuvor aus dem Ordner Ergebnisordner der Resultate einer Session gefiltert wurden (Zeilen 152 bis 164). 

Nach erfolgreicher Erstellung ist ein Video in der Gesamtlänge der Testsession vorhanden und beinhaltet alle Touch-Eingaben am Client. 
Das Bild eines Touches wird solange angezeigt bis ein neuer Touch-Screenshot vorhanden ist. 

\begin{lstlisting}[label=node-imgConcat,language=Java, caption=Screenshot-Videos zu einem Videos konkatenieren, firstnumber=149]

// ...

files.forEach(function (file) {

	if(file != '.DS_Store' && file != 'result.mpg') {

		if (pushed > 0) {
			videopaths += "\|"+dir+file;
		} else {
			videopaths += dir+file;
		}
		
		pushed++;
	}
});

var imgoutput = dir + 'result.mpg';
//params here
var params  = [
	'-i', 'concat:'+videopaths,
	'-c', 'copy',
	'-y',
	imgoutput
];

var stream = avconv(params);

// ...
\end{lstlisting}

In den folgenden Schritten wird das Gesamtergebnis vorbereitet und erstellt. 

Im Listing \ref{node-videoPad} wird die das Verdoppeln der Videobreite implementiert. 
Dies ist notwendig, damit das Endergebnis des Videos die richtige Größe hat und die zusammengefügten Screenshots direkt neben dem aufgenommenen Video platziert werden können. 

Der Parameter \textbf{filter:v} erhält als Werte drei mal \textbf{transpose=1}, damit wird die Eingabe 270 Grad (im Uhrzeigersinn) gedreht, da es um 90 Grad im Uhrzeigersinn gedreht am Server ankommt. Negative Werte werden von \texttt{transpose} nicht akzeptiert. Der zweite Parameterwert \emph{pad=2*iw} ergänzt die Videobreite um zwei mal der Eingabebreite (\emph{iw}, input width).

\begin{lstlisting}[label=node-videoPad,language=Java, caption=Erstellung der Ergebnisgrundlage, firstnumber=196]
//...

var output = './uploads/'+user+'/'+session+'/results/pad.mpg';
//params here
var params  = [
	'-i', videopaths,
	'-filter:v', 'transpose=1, transpose=1, transpose=1, pad=2*iw',
	'-strict', 'experimental',
	'-y',
	output
];

var stream = avconv(params);

// ...
\end{lstlisting}

Zum Schluss wird das Gesamtvideo erstellt, dessen Implementierung im Listing \ref{node-videoResult} zu sehen ist. 
Als Eingabe wird das zuvor erstellte Video \emph{pad.mpg} genutzt und das Ergebnisvideo der Screenshots wird als \texttt{overlay}-Filter (Zeile 214) ab dem X-Wert 720 auf das Eingabevideo gelegt.

\begin{lstlisting}[label=node-videoResult,language=Java, caption=Gesamtvideo kombinieren, firstnumber=211]
// ...

var input = output;
var filter ="movie="+ imgoutput +" [wm];[in][wm] overlay=720:0 [out]";
var finaloutput = './uploads/'+user+'/'+session+'/results/result.mpg';
var params = [
	'-i', input,
	'-vf', filter,
	'-strict', 'experimental',
	'-y',
	finaloutput
];

var stream = avconv(params);

stream.pipe(process.stdout);

stream.once('end', function(exitCode, signal) {
	console.log('all fin '+ exitCode );

	var qstr = 'INSERT INTO results (file, timestamp, sessions_idsessions) VALUES ' +
			'("'+ finaloutput + '", NOW(),' + session + ')';

	var query = new Query;

	query.execute(qstr, function(rows) {
		console.log(rows);
	});
});

// ...
\end{lstlisting}

\subsection{Server-Fazit \label{sec:server-fazit}}

Die Serverkomponente implementiert keine Sicherheitsmaßnahmen und wurde weder Lasttests unterzogen, noch wurden alle Funktionen auf Korrektheit überprüft.

Die Anwendung wurde soweit entwickelt, dass der Android-Client damit kommunizieren kann und alle Funktionen des Clients bedienen kann.
Weiterhin können die aufgezeichneten Daten zu einem Gesamtergebnis konvertiert werden.