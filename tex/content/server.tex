\section{Server}
\label{server}
Wie in Abschnitt \ref{app_architecture} beschrieben, müssen die aufgezeichneten Daten der Clientbibliothek weiter verarbeitet werden. % nochmal etwas genauer

Im Folgenden wird die Struktur des Server-Moduls beschrieben, welches mit NodeJS\footnote{NodeJS: } umgesetzt wurde.
%TODO: footnote ++ update

%TODO: subsection title ?!
%TODO: Vagrant verfassen 
\subsection{Virtualisierung mit Vagrant}
Um die Bereitstellung der Serverumgebung zu vereinfachen wurde die virtuelle Maschine mittels Vagrant konfiguriert. 

\subsection{NodeJS HTTP-Server und Middleware}

\begin{lstlisting}[label=minimal_node_http_server,language=Java, caption=Minimaler Node-HTTP-Server]
/**
 * Module dependencies.
 */
var express = require('express');
var routes = require('./routes');
var http = require('http');
var path = require('path');
var app = express();

// environments config
app.set('port', process.env.PORT || 3000);
app.set('views', path.join(__dirname, 'views'));
app.set('view engine', 'ejs');
app.use(express.favicon());
app.use(express.logger('dev'));
app.use(express.json());
app.use(express.urlencoded());
app.use(express.methodOverride());
app.use(app.router);
app.use(express.static(path.join(__dirname, 'public')));

// simple test route
app.get('/ping', routes.pong);


// server take off
http.createServer(app).listen(app.get('port'), function(){
  console.log('Express server listening on port ' + app.get('port'));
});
\end{lstlisting}



\subsection{Datenbank Backend}

Zur persistenten Speicherung der Nutzdaten wird MySQL verwendet. MySQL bietet sich an, da es Verbreitung %TODO: Links in footnotes

\subsection{APIs}

\subsection{Videoproduktion}

