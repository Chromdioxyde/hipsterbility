\section{Android-Client}

In diesem Abschnitt wird die Entwicklung der Android-Bibliothek beschrieben. Zur Entwicklung wurde die IDE \textit{IntelliJ Idea}\footnote{IntelliJ Idea: http://www.jetbrains.com/idea/} verwendet. Screenshots und Anleitungen beziehen sich diesbezüglich auf diese Software. Konfigurationen und Einstellungen können sich zu anderen IDEs unterscheiden. 

Die Quellen zu den Projekten befinden sich im Github-Repository im Unterordner \textit{client}.

\subsection{Bibliotheksmodul}

Zur Entwicklung der Bibliothek wurden zwei Projekte verwendet - Einmal das Projekt für die Bibliothek selbst und weiterhin ein Android-Projekt zum Testen der Funktionen während der Entwicklung. 

\subsubsection{Monolithischer Wrapper}

\subsection{Einbinden der Bibliothek in eigene Anwendungen}

\subsubsection{Dynamisches Einbinden - Pre-Dexing}

Beim Pre-Dexing können mehrere Android-Projekte zu einem Projekt hinzugefügt werden. 
% TODO: updating

Da die Bibliothek neueste Funktionen des Android-Betriebssystems verwendet, muss auf dem  Gerät zur Kompilierung einer Anwendung aktuelle Build-Tools der Android-SDK vorliegen. Im Falle der Bibliothek ist dies API-Level 19. Offenbar hat sich hier ein Bug eingeschlichen und es ist unter der Version 19.0 kein Pre-Dexing möglich. Dies wurde allerdings bereits behoben. Es sei gesagt, dass mindestens die Version 19.0.1 verwendet werden muss um Pre-Dexing durchführen zu können. 
% TODO: update, wichtig? Screenshot? 

\subsubsection{Statisches Einbinden als JAR-Archiv}
