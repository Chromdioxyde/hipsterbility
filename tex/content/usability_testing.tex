\section{Testen von Usability}
\label{usability_testing}

Dieser Abschnitt stellt mögliche Vorgehen beim Testen von Usability vor, grenzt sie voneinander ab und bewertet sie für die Implementierung der Bibliothek. 

Die Anforderungen an das \textit{Mobile Web} und an \textit{Apps} unterscheiden sich stark von den klassischen Desktop-Anforderungen. 
Aufgrund der geringeren Displaygröße und Multitouch-Funktionalitäten in der Bedienung lösen neue Anforderungen die klassischen ab oder erweitern die vorhanden. 
Ebenfalls wichtig sind die Zielgruppen, sowie der Nutzungskontext einer App. 

\subsection{Testmethoden \label{sec:testmethoden}}

\subsubsection{Formale Usability-Tests \label{sec:formal-usability-test}}

Nach dem Ablauf formaler Usability-Tests, kann eine Anwendung hinsichtlich Usability formativ und summativ evaluiert werden. 
Während die formative Evaluation die Verbesserung des zu prüfenden Systems zum Ziel hat, soll die summative Evaluation eine zusammenfassende Qualitätskontrolle sicherstellen. Der allgemeine Ablauf ist folgender: es wird eine Testserie mit sog. \textit{Standardaufgaben} erstellt, wobei dies in der aus Sicht des Anwenders formuliert werden sollte. % TODO cite
Weiterhin werden Rahmenbedingungen ausgehandelt. 

Der Test selbst findet in einem dafür vorgesehenem Testraum statt. 
Die Beobachter greifen nur ein, wenn es notwendig ist. 

Nach dem Test wird der Testbericht festgehalten und es können die Resultate analysiert werden. 
% TODO Cite

\subsubsection{Usability Walkthrough}

Als Alternative zum formalen Usability-Test kann der \textit{Usability-Walkthrough} verwendet werden. 
Unterschied zum formalen Usability-Test ist dass Testpersonen nicht unter Beobachtung gestellt werden. 
Stattdessen wird der gesamte Test moderiert und aktiv begleitet. 
Testpersonen können während des Tests Fragen stellen und auf bestimmte Situationen reagieren und der Testbegleiter muss wissen wie Testnutzer angeleitet werden können. % TODO cite

\subsubsection{Usability-Feld-Tests}

Das \textit{mobile Usability Lab} nutzt mobile Geräte, wie etwa Notebooks oder Webcams aus dem Gebrauch des Testnutzers, anstatt Laborausstattung zu verwenden. 
Eingesetzt wird das mobile Labor, sobald Benutzer unter definierten Bedingungen an einem bestimmten Ort getestet werden sollen. 
Für gewöhnlich dort, wo der Arbeitsplatz einen hohen Einfluss auf die Benutzung hat. 
Zum Beispiel Bankautomaten an öffentlichen Plätzen. 
Die Ausstattung ist dementsprechend günstiger. % TODO cite

\subsection{Exkurs: Thinking aloud\label{sec:thinking_aloud}}
Neben den zuvor genannten Methoden gehört auch die \emph{Thinking aloud} zu den gängigen Usability-Tests.
Bei der Methode soll die Testperson laufend die Gedanken aussprechen die ihr gerade in den Sinn kommen, während sie das zu testende System benutzt.
Dies führt dazu, dass fehlerhafte Konzepte und Fehleinschätzungen identifiziert werden können, indem die Interpretation des Systems durch die Testperson ausgewertet wird.
Dabei werden besonders die problematische Bereiche in Benutzeroberflächen herausgestellt.
Wichtig bei dieser Testmethode ist die Zeit, denn das Ziel ist es, besonders die frischen Eindrücke der Testpersonen zu erhalten, da das Gedächtnis die Eindrücke mit der Zeit verfälschen kann.

Als Variante kann auch noch eine zweite Testperson herangezogen werden, da es verbale Äußerungen im Beisein anderer Personen natürlicher wirken können.
Durch konstruktive Interaktion der Testbenutzer können so noch weitere hilfreiche Kommentare erfolgen, welche bei einer einzelnen Testperson evtl. nicht ausgesprochen würden \cite[vgl.][73\psq]{usability_engineering_software_developers}.


\cite{usabilityblog_wasBeachten}

\cite{usabilityblog_eResult}

\subsection{Auswertung}

Mit Hilfe des formalen Usability-Test aus dem Abschnitt \ref{sec:formal-usability-test} lassen sich Schwachstellen von Benutzeroberflächen und Schwierigkeiten in der Anwendung erkennen.
Allerdings können derartige Tests nicht ständig durchgeführt werden, da sie aufwändig in der Vorbereitung sind und finanziert werden müssen. 
Testnutzer, Labor und Durchführung benötigen Zeit und Geld. 
Resultierend darin kommt der Usability-Test höchstwahrscheinlich erst spät zum Einsatz, wodurch Testergebnisse vielleicht nicht umgesetzt werden können, da das Projektbudget aufgebraucht ist oder der Zeitplan nicht mehr eingehalten werden kann. 

Usability Walkthroughs könnten in einem frühen Projektstadium eingesetzt werden um bereits Schwächen am unfertigen Produkt zu erkennen und ggf. zu beseitigen. 
Auch bei diesem Test bedarf es Vorbereitung, Testpersonen und Kosten.

Im mobilen Labor, bzw. Feldtest, können ebenfalls hohe Kosten entstehen. 
Hardware für den Einsatz an bestimmten Orten wird benötigt. 
Dies kann unter Umständen sehr spezielle, bzw. teure Ausstattung sein. 
Mit den integrierten Hardwaremodulen eines Mobilgeräts kann der Einsatz genauer bestimmt werden und ist damit unverfälscht.

Nach den vorgestellten formalen Usability-Testmethoden können drei Hauptaspekte extrahiert werden: 
\begin{itemize}
	\item{Kostenaspekt}
	\item{Zeitaspekt}
	\item{Geräteaspekt}
\end{itemize}

Die Verwendung von teurer Laborausstattung, der Einsatz mehrerer Testbeobachter und vieler Testnutzer führt zu hohen Kosten bei Usability-Tests. 
Die aktuelle Ausstattung von mobilen Endgeräten ist qualitativ ausreichend hoch und steht im Vergleich zu Laborausstattung günstig zur Verfügung. 

Usability-Walkthroughs können auch ohne Anleitung durchgeführt werden. 
Die Sensoren, zum Beispiel Bild und Ton, müssen der zu testenden Anwendung zur Verfügung stehen und gesteuert werden können. 
Wird der Geräteaspekt betrachtet, so stehen alle Sensoren auf aktuellen Endgeräten zur Verfügung.

Die Kosten für Vorbereitungen von Tests müssen nicht bei jedem Testvorgang neu erstellt werden. Ein geeignetes Werkzeug würde beispielsweise eine Liste von Standardaufgaben für eine Anwendung zu sichern, sodass diese zu einem späteren Zeitpunkt wiederholt werden können (Zeit- und Kostenaspekt). 

Mit der aktuellen Hardware mobiler Endgeräte lassen sich Tests semi-automatisieren, indem ein Rahmenwerk für die Sensoren geschaffen wird, mit deren Hilfe sich die verschiedenen Eingabemedien ohne zusätzliche Ausstattung aufzeichnen und analysieren lassen. 
So kann der Testprozess während der Entwicklung genutzt werden. 
Entweder von Testnutzern, dem Projektmanagement oder gar von Entwicklern selbst. 

Die Anschaffung evtl. teurer Laborausstattung entfällt, da die Sensoren im mobilen Endgerät vorhanden sind und vom Rahmenwerk genutzt werden können. 

\subsubsection{Zielsetzung \label{sec: zielsetzung}}

Im Rahmen dieser Arbeit wurde ein Framework, bzw. Programmierbibliothek entwickelt, welches die Schwächen des Usability-Tests umgeht. 
Vor allem der Einsatz zusätzlicher Geräte entfällt, da die integrierten Module (Kamera, Mikrofon, Touch-Display) verwendet werden. 
Zudem kann mittels einer Programmierbibliothek der Einsatz von Usability-Tests in den Entwicklungsprozess integriert werden. 
So können ggf. Schwächen an der Benutzeroberfläche schon früh durch Entwickler erkannt werden.

Da Zugriff auf diverse Hardware erfolgt muss dies vom System erlaubt sein. 
Lediglich die offene Natur des Android-Betriebssystems ermöglicht es ohne Umwege, wie etwa Jailbreaks, Sensoren des mobilen Endgeräts zu steuern. 
Folglich wird die Programmierbibliothek nur für Android als native Anwendungsbibliothek entwickelt. 

Neben dem Aufzeichnen von Medien, Gesten und Touch-Eingaben müssen diese weiter verarbeitet werden. 
Es ist unvorteilhaft die Verarbeitung am mobilen Geräten durchzuführen, da die zu testende Anwendung oder das gesamte System unresponsiv werden könnten. 
Stattdessen wurde eine Client-Server-Architektur implementiert, welche über HTTP mit einer RESTful-API kommuniziert. 

