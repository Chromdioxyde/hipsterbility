\section{Testen von Usability}
\label{usability_testing}

Dieser Abschnitt stellt mögliche Vorgehen beim Testen von Usability vor, grenzt sie voneinander ab und bewertet sie für die Implementierung der Bibliothek. 

Die Anforderungen an das \textit{Mobile Web} und an \textit{Apps} unterscheiden sich stark von den klassischen Desktop-Anforderungen. Aufgrund der geringeren Displaygröße und Multitouch-Funktionalitäten in der Bedienung lösen neue Anforderungen die klassischen ab oder erweitern die vorhanden Anforderungen. 

\subsection{Testmethoden \label{sec_testmethoden}}

% TODO: Testarten, Methoden, Abgrenzung, .... Theorie und Transferleistung
\subsubsection{Formale Usability-Tests}

Nach dem Ablauf formaler Usability-Tests, kann eine Anwendung hinsichtlich Usability formativ und summativ evaluiert werden. Während die formative Evaluation die Verbesserung des zu prüfenden Systems zum Ziel hat, soll die summative Evaluation eine zusammenfassende Qualitätskontrolle sicherstellen. 
Der allgemeine Ablauf ist folgernder: es wird eine Testserie mit sog. Standardaufgaben erstelllt. 

\subsubsection{Usability Walkthrough}
Neben dem formalen Usability-Test in einer kontrollierten Umgebung kann ein Test auch durch durch den Testleiter moderiert werden. Ebenso wie beim formalen Usability-Test wird hier der 
\subsubsection{Mobiles Usability Lab / Usability Lab}



\cite{usabilityblog_wasBeachten}

\cite{usabilityblog_eResult}