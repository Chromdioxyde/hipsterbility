\section{Ergebnis und offene Punkte}

Das Ergebnis der Projektarbeit ist ein Framework, bestehend aus Prototypen-Applikationen.
Es soll das Durchführen von Usability-Tests stark vereinfachen, indem keine Aufwändige Ausrüstung benötigt wird, sondern lediglich Mobiles Endgerät mit Android 4.4 und PC bzw. eine \ac{VM}, die als Server dient.
Die vorliegende Implementierung ist nicht für den produktiven Einsatz bereit, jedoch wurden alle Kernpunkte bearbeitet und fragliche Funktionen erfolgreich implementiert.

Damit kann das Ergebnis dieser Arbeit als Startpunkt für weiterführende Arbeiten und Untersuchungen dienen.
Mit entsprechendem Arbeitsaufwand ist auch eine ausreichende Reife für den Produktiveinsatz erreichbar.

Dazu müssten allerdings noch einige offene Punkte bearbeitet werden.
Das Webinterface für die Verwaltung ist nur skizziert und Sicherheitsfunktionen wurden nur rudimentär implementiert. 
Hier sollte eine sichere Authentifizierung verwendet werden und der vollständige Funktionsumfang müsste hergestellt werden.
Allen Komponenten sollten die fehlenden Objekte aus dem Datenmodell mit entsprechender Funktionalität hinzugefügt werden.
Auch die Android-Bibliothek benötigt einige Iterationsstufen, bis sie vollkommen einsatzbereit ist (siehe Abschnitt \ref{sec:client_offene_punkte}).

Das Gesamtziel wurde jedoch erreicht und darauf kann in Zukunft aufgebaut werden.