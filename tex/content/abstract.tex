
\begin{abstract}

\noindent\textbf{Zusammenfassung:}\\ \\
Die vorliegende Ausarbeitung wurde in \LaTeX verfasst und ist eine gemeinsame Arbeit von Albert Hoffmann und Oliver Erxleben an der Hochschule Osnabrück / University of Applied Sciences im Masterstudiengang Informatik - Mobile und Verteilte Anwendungen des Fachbereichs Ingenieurswissenschaften und Informatik für das Modul Mensch-Maschine-Kommunikation im Wintersemester 2013/14. 

Die Arbeit dokumentiert die Entwicklung einer Programmierbibliothek zum Test der Usability von Mobilanwendungen für das Android-Betriebssystem.

Das Ziel ist es, eine Bibliothek für das Android Betriebssystem zu schreiben, die Entwickler einfach in ihre Anwendungen integrieren können und mit der sich Usability Untersuchungen durchführen lassen.

Zur Verwaltung wird für Entwickler/Testleiter eine webbasierte Oberfläche bereitgestellt, über die Test und Aufgaben verwaltet werden können. 
Das Ergebnis ist ein Video, welches das Kamerabild der Gerätefrontkamera und den Bildschirminhalt des Testgerätes enthält. Diese Video kann über die Verwaltungssoftware, nach Abschluss der Tests, Angezeigt werden.

Für die Kommunikation wird außerdem eine \ac{API} entwickelt, welche von der Verwaltungsoberfläche und der Android Bibliothek genutzt wird.
Alle Komponenten werden als Prototypen entwickelt und sind eher als Proof-of-Concept anzusehen, entsprechend eignen sich die frühen Versionen weniger für den Produktiveinsatz.

\end{abstract}