\section{Motivation}
\label{motivation}

Mobile Anwendungen, sowie mobile Versionen einer Web-Site oder Web-Anwendung sind seit dem Aufkommen von Smartphones und Tablets im täglichen Leben unserer Gesellschaft verankert. Diese Anwendungen, allgemeinhin als \textit{Apps} bezeichnet, sollen uns im Alltag helfen, vernetzen, navigieren oder gar zerstreuen. Die Möglichkeiten der mobilen Hardware erweitern sich mit jeder Revision. Neue Programmierschnittstellen und präzisere Sensoren ermöglichen eine stetige Verbesserung und neue Nutzungsmöglichkeiten für mobile Endgeräte. 

Neben den Funktionen die durch Apps bereitgestellt werden, müssen diese aber auch leicht zu benutzen sein. Jede Funktion kann noch so hilfreich sein, wenn sie dabei nur schlecht bedienbar ist hilft sie dem Nutzer nicht.

Das Thema \textit{Usability} spielt für die Software-Entwicklung eine immer größere Rolle. Eine Anwendung kann sich nur gut verkaufen, wenn sie auch bedienbar oder benutzerfreundlich ist, wenn sie sich "richtig anfühlt". Usability ist aber auch kein neues Thema im Software-Engineering. Seit Jahren beschäftigen sich Menschen mit der Verbesserung der Usability von Anwendungen.

Es werden Labor- oder Feldtests durchgeführt, Dienstleistungen werden angeboten und es werden Testprotokolle aufgezeichnet. Beispiele für solche Dienstleister sind: 

\begin{itemize}
    \item{http://www.userfeedbackhq.com/}
    \item{http://rapidusertests.com/}
    \item{http://www.usertesting.com/mobile}
\end{itemize}

Die Dienstleister erstellen mit Testsubjekten Anwendungstests und geben ein Feedback in Form von Video-, Audio- und Textmaterial. Dazu werden unterschiedliche Techniken eingesetzt. Zum Beispiel die Aufnahme des Smartphones über eine Helmkamera. 

Die gelisteten Dienste prüfen eine fertige App, einen Prototyp oder einen bestimmten Entwicklungsstand einer Anwendung. Usability-Tests sind aufwändig und teuer. Je nach Zielgruppe der Anwendung können hohe Kosten für einen Test entstehen. Es müssen geeignete Tester gefunden und es muss unter Umständen Hardware beschafft werden. Zudem kann das Testen sehr zeitintensiv werden. Weitere oder sehr hohe Kosten können entstehen, wenn der Usability-Test negativ ausfällt und viele Teile einer vielleicht fertigen Anwendung neu konzeptioniert und entwickelt werden müssen. 

Der aktuelle Stand der Technik zeigt das die Hardware des Endgeräts bereits viele technischen Vorraussetzungen erfüllt um Usability-Tests durchzuführen. So besitzen beispielweise fast alle Smartphones eine Frontkamera und Mikrofon womit Bild und Ton eines Testers aufgezeichnet werden kann. 

Wenn Usability-Tests bereits früh durchgeführt werden können, Funktionen zum Test von Usability bereitgestellt werden und kleinere Pakete der Software getestet werden, können schon früh und rechtzeitig Probleme bei der Usability gefunden und behoben werden.

Es stellt sich die Frage, wie Usability-Testing bereits während der Entwicklung eingesetzt werden kann und wie eine Bibliothek, bzw. ein Framework, Funktionen zum Usability-Testing bereitstellen kann. Welche Funktionen muss eine Bibliothek aufweisen um Usability zu testen? Diesen Fragen, sowie der Machbarkeit einer solchen Programmierbibliothek, widmet sich die vorliegende Ausarbeitungen. 

%TODO: Revision 1 - Korrektur?!

%TODO ODL 

%TODO example citings - entfernen

\cite{usabilityEngineeringKompakt}

\cite{nodejs_therightway}